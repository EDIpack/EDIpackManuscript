\section{Interoperability}
The recent growing availability of state-of-the-art software dedicated
to the solution of quantum impurity problems using different methods
poses a serious challenge to test accuracy and realiability of the
results.
As such, complex softwares are requested to develop a higher level of
interoperability, i.e. the possibility to operate with other software
possibly using different programming language.
Modern Fortran, which is the language of choice for \NAME, since many
years supports the standardized generation of procedures and
variables that are interoperable with C.


\subsection{C-bindings}\label{sSecCbind}
The interoperability with C language is encoded in the
{\tt ISO\_C\_BINDING} module part of Fortran
standard since 2003. The module contains definitions of named
constants, types and procedures for C interoperability.
Alongside, the second key feature essential to expose to C any Fortran
entity is the {\tt BIND(C)} intrinsic function.



\subsection{EDIpy, the Python API}\label{SecPyAPI}

\subsubsection{Build from source}
EDIpy2 is available as a stand-alone
module which depends on both \NAME and SciFortran. The package can be
obtained from the repository
\href{https://github.com/EDIpack/EDIpy2.0}{EDIpy2}.

\begin{lstlisting}[style=mybash]
git clone https://github.com/edipack/EDIpy2 EDIpy2
cd EDIpy2
pip install . 
\end{lstlisting}
In some more recent Python distribution the flag {\tt
  --break-system-packages} might be required to complete
installation or a virtual environment should be used instead. 

\subsubsection{Anaconda}
As for EDIpack2, also the Python API in EDIpy2 are available through
Anaconda packaging. In this case the resolution of the dependencies is
taken care from Conda itself:

\begin{lstlisting}[style=mybash]
conda create -n edipack
conda activate edipack
conda install -c conda-forge -c edipack edipack2
\end{lstlisting}





\subsection{TRIQS interface}\label{SecEDI2Triqs}
A purely Python \NAME to Triqs interface is available, leveraging on
the C-bindings and Python API. The corresponding module depends on
\NAME (which ultimately depends on SciFortran) and Triqs.
Assuming the two software are correctly installed in the OS, the
EDIpack2Triqs interface is installed as follows:

\begin{lstlisting}[style=mybash]
git clone https://github.com/krivenko/edipack2triqs
cd edipack2triqs
pip install .
\end{lstlisting}



\subsection{W2Dynamics interface}\label{SecEDI2W2Dyn}




\subsection{EDIjl, the Julia API}\label{SecJlAPI}
