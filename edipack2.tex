%% Use the options 1p,twocolumn; 3p; 3p,twocolumn; 5p; or
%% 5p,twocolumn, preprint, review
\documentclass[preprint,3p,10pt]{elsarticle}

%%%%%%%%%%%%%%%%%%%%%%%%%%%%%%%%%%%%%%%%%%%%%%%%%%%%%%%

\usepackage[english]{babel}
\usepackage{amsmath,amssymb,amsfonts,amsthm}
\usepackage{makecell}
\usepackage{array}
\usepackage{stackengine}
\newlength\llength
\llength=1.38ex\relax
\usepackage[dvipsnames]{xcolor}
\usepackage{makecell}
\usepackage{subfiles}
\usepackage{comment}
\newcommand{\fixme}[1]{{\color{BrickRed} #1}}

%% natbib.sty is loaded by default. natbib options with \biboptions{...}
\biboptions{sort&compress,super}

% Hypelinks in the document; settings
\usepackage[colorlinks=true,linkcolor=blue,citecolor=red]{hyperref}
% \usepackage[normalem]{ulem}


\usepackage{listings}
\lstdefinestyle{fstyle}{
  frame=lines,
  language=fortran,
  basicstyle=\footnotesize,
  stringstyle=\ttfamily,
  commentstyle=\itshape,
  fontadjust=true,
  keywordstyle=\color{red},
  % morekeywords={*,...},
  mathescape,
  numbers=left, numberstyle=\tiny, stepnumber=1, numbersep=3pt
}

\lstdefinestyle{mypython}{
  frame=lines,
  language=python,
  basicstyle=\footnotesize,
  stringstyle=\ttfamily,
  commentstyle=\itshape,
  fontadjust=true,
  keywordstyle=\color{red},
  % morekeywords={*,...},
  mathescape,
  numbers=left, numberstyle=\tiny, stepnumber=1, numbersep=3pt
}

\definecolor{mymauve}{rgb}{0.58, 0, 0.82}
\lstdefinestyle{mybash}{
  frame=lines,
  language=bash,
  basicstyle=\footnotesize,
  captionpos=b,
  stringstyle=\color{mymauve},%\ttfamily,
  commentstyle=\itshape,
  fontadjust=true,
  keywordstyle=\bfseries,
  morekeywords={*,git,mkdir,cmake,make},
  mathescape,
  numbers=left, numberstyle=\tiny, stepnumber=1, numbersep=3pt
}

\lstdefinestyle{cstyle}{
  frame=lines,
  language=c,
  basicstyle=\footnotesize,
  stringstyle=\ttfamily,
  commentstyle=\itshape,
  fontadjust=true,
  keywordstyle=\color{red},
  % morekeywords={*,...},
  mathescape,
  numbers=left, numberstyle=\tiny, stepnumber=1, numbersep=3pt
}

% % MINTED:
% % \usepackage[cache=true,cachedir=minted-cache]{minted}
% % \usepackage[finalizecache=true,cachedir=minted-cache]{minted}
% \usepackage[frozencache=true,cachedir=minted-cache]{minted}
% \setminted[fortran]{linenos,mathescape,frame=lines,style=bw,framesep=1mm,
%   baselinestretch=1,fontsize=\footnotesize}



\usepackage{array}
\usepackage{tabularx}
\usepackage{ltablex}

\DeclareMathAlphabet\mathbfcal{OMS}{cmsy}{b}{n}


\renewcommand{\arraystretch}{1.4}
% \newcolumntype{T}[1]{>{\tt\footnotesize}m{{#1}}}
% \newcolumntype{D}[1]{>{\it\footnotesize}m{#1}}
% \newcolumntype{M}[1]{>{\scriptsize}m{#1}}

\newcolumntype{T}[1]{>{\tt\footnotesize\raggedright\arraybackslash}p{#1}}
\newcolumntype{D}[1]{>{\it\footnotesize\raggedright\arraybackslash}p{#1}}
\newcolumntype{M}[1]{>{\scriptsize\raggedright\arraybackslash}p{#1}}


\newcommand{\onlinecite}[1]{\nocite{#1}\!\!\!\citenum{#1}}

\newcommand {\note}[1]{{\color{blue} [{\bf NOTE}: \bf #1]}}
\newcommand {\aac}[1]{{\color{red} [{\bf AA}: \bf #1]}}
\newcommand {\new}[1]{{\color{blue}\it #1}}
%\newcommand {\new}[1]{{#1}}
% \DeclareMathAlphabet\mathbfcal{OMS}{cmsy}{b}{n}

\usepackage[inline]{showlabels}


%Reference to a given labelled equation
%and definition of a bib. element.
%-------------------------------------------
\newcommand{\equ}[1]
{Eq.\,(\ref{#1})}

\newcommand{\figu}[1]
{Fig.\,\ref{#1}}

\newcommand{\secu}[1]
{Sec.\,\ref{#1}}

\newcommand{\ket}[1]
{|#1\rangle}

\newcommand{\bra}[1]
{\langle #1|}

\newcommand{\sgn}
{\mathop{\mathrm{sgn}}}



%SIMBOLI VARI
%%%%%%%%%%%%%%%%%%%%%%%%%%%%%%%%%%%%%%%%%%%%%%%%%%%%%%%
\def\bcen{\begin{center}}
\def\ecen{\end{center}}

\def\a{\alpha}       \def\b{\beta}   \def\g{\gamma}   \def\d{\delta}
\def\e{\varepsilon}  \def\z{\zeta}   \def\h{\eta}     \def\th{\theta}
\def\k{\kappa}       \def\l{\lambda} \def\m{\mu}      \def\n{\nu}
\def\x{\xi}          \def\p{\pi}     \def\r{\rho}     \def\s{\sigma}
\def\t{\tau}         \def\f{\varphi} \def\ph{\varphi} \def\c{\chi}
\def\ps{\pi}        \def\y{\upsilon}\def\o{\omega}   \def\si{\varsigma}
\def\G{\Gamma}       \def\D{\Delta}  \def\Th{\Theta}  \def\L{\Lambda}
\def\X{\Xi}          \def\P{\Pi}     \def\Si{\Sigma}  \def\F{\Phi}
\def\Ps{\Psi}        \def\O{\Omega}  \def\Y{\Upsilon} \def\lg{\langle}

\def\PP{{\cal P}}\def\EE{{\cal E}}\def\MM{{\cal M}} \def\VV{{\cal V}}
\def\CC{{\cal C}}\def\FF{{\cal F}}\def\HH{{\cal H}}\def\WW{{\cal W}}
\def\TT{{\cal T}}\def\NN{{\cal N}}\def\BB{{\cal B}} \def\II{{\cal I}}
\def\RR{{\cal R}}\def\LL{{\cal L}}\def\JJ{{\cal J}} \def\OO{{\cal O}}
\def\DD{{\cal D}}
\def\AA{{\cal A}}
\def\GG{{\cal G}} \def\SS{{\cal S}}
\def\ZZ{{\cal Z}} \def\UU{{\cal U}}
\def\SB{{\cal S}{\cal B}}
\def\aa{{\V \a}}
\def\hh{{\V h}}\def\HHH{{\V H}}
%\def\AA{{\V A}}
%\def\GG{{\V G}}\def\BB{{\V B}}\def\aaa{{\V a}}\def\bbb{{\V b}}
\def\nn{{\V \n}}\def\pp{{\V p}}\def\mm{{\V m}}\def\qq{{\bf q}}
\def\RRR{\mathbb{R}} \def\CCC{\mathbb{C}} \def\NNN{\mathbb{N}}
\def\ZZZ{\mathbb{Z}}
%\def\TTT{\hbox{\msytw T}}



\def\ul{\underline}
\def\=={\equiv}
\def\defi{{\buildrel def \over =}}
\def\lft{\left} \def\rgt{\right} \def\dpr{\partial} \def\der{{\rm d}}
\def\us{\underline \s} \def\ue{{\underline \e}}
\def\la{\left\langle}
\def\ra{\right\rangle}
\def\qed{\raise1pt\hbox{\vrule height5pt width5pt depth0pt}}
\def\iome{i\omega_n} \def\iom{i\omega} \def\iom#1{i\omega_{#1}}
\def\iomn{i\omega_n}
\def\epsk{\epsilon({\bf k})} \def\Ga{\Gamma_{\alpha}}
\def\Seff{S_{eff}}  \def\dinf{$d\rightarrow\infty\,$}
\def\cG0{{\cal G}_0}
\def\cG{{\cal G}}  \def\cU{{\cal U}}  \def\cS{{\cal S}}
\def\spinup{\uparrow} \def\spindown{\downarrow} \def\spindw{\downarrow}
\def\up{\uparrow} \def\down{\downarrow} \def\dw{\downarrow}


\def\Ak{{\bf A}} \def\Akt{{\bf A}(t)} \def\Ek{{\mathbf E}}
% \def\Im{\mbox{Im}}
\def\=={\equiv}
\def\defi{{\buildrel def \over =}} \def\nt{\widetilde{n}}
\def\Im{{\rm Im}} \def\Re{{\rm Re}} \def\Tr{{\rm Tr}}
\def\det{{\rm det}\,} 


\def\ibra{\langle}
\def\iket{\rangle}

\def\ka{{\bf k}}
\def\vk{{\bf k}}
\def\qa{{\bf q}}
\def\vQ{{\bf Q}}
\def\vr{{\bf r}}
\def\q{{\bf q}}
\def\R{{\bf R}}
\def\vR{{\bf R}}
\def\kx{{ k_x}}
\def\ia{{\bf i}}
\def\ja{{\bf j}}

\usepackage{bbold}
\def\11{\mathbb{1}}
\def\00{\mathbf{0}}
\def\NAME{{\rm EDIpack2 }}

%%%%%%%%%%%%%%%%%%%%%%%%%%%%%%%%%%%%%%%%%%%%%%%%%%%%%%% 


 
\journal{Computer Physics Communications}

\begin{document}

\begin{frontmatter}

\title{EDIpack2: interoperable Lanczos-based solver for generic quantum impurity problems}
\author[a,b,c]{L.~Crippa}
\author[a]{I.~Krivenko}
\author[d]{S.~Giuli}
\author[d]{G.~Bellomia}
 \author[c]{A.~Kowalski}
\author[e]{F.~Petocchi}
 \author[f]{M.~Wallerberger}
 \author[g]{G.~Mazza}
 \author[h]{L.~de Medici}
 \author[c,b]{G.~Sangiovanni}
 \author[d,i]{M.~Capone}
\author[i]{A.~Amaricci}
% \author[]{A.~Scazzola}

% \cortext[author] {Corresponding author.\\\textit{E-mail address:} amaricci@iom.cnr.it}
\newcommand{\CNRIOM}{CNR-IOM, Istituto Officina dei Materiali,
  Consiglio Nazionale delle Ricerche, Via Bonomea 265, 34136
  Trieste, Italy}
\newcommand{\SISSA}{Scuola Internazionale Superiore di Studi Avanzati (SISSA),
  Via Bonomea 265, 34136 Trieste, Italy}
\newcommand{\ITPHamburg}{I. Institute of Theoretical Physics,
  University of Hamburg, Notkestrasse 9, 22607 Hamburg, Germany}
\newcommand{\WZBURG}{Institut f\"ur Theoretische Physik und
  Astrophysik,Universit\"at W\"urzburg, 97074 W\"urzburg, Germany}
\newcommand{\CTQMAT}{W\"urzburg-Dresden Cluster of Excellence ct.qmat, 01062 Dresden, Germany}
\newcommand{\Geneve}{Department of Quantum Matter Physics, University of
  Geneva, Quai Ernest-Ansermet 24, 1211 Geneva, Switzerland}
\newcommand{\UPISA}{Department of Physics ``E. Fermi'' University of
  Pisa, Largo B. Pontecorvo 3, 56127 Pisa, Italy}
\newcommand{\ESPCI}{LPEM, ESPCI Paris, PSL Research University, CNRS, Sorbonne Universit\'e, 75005 Paris, France}
\newcommand{\TUW}{Institure of Solid State Physics, TU Wien, 1040 Vienna, Austria}

\address[a]{\ITPHamburg}
\address[b]{\CTQMAT}
\address[c]{\WZBURG}
\address[d]{\SISSA}
\address[e]{\Geneve}
\address[f]{\TUW}
\address[g]{\UPISA}
\address[h]{\ESPCI}
\address[i]{\CNRIOM}
% \address[]{Politecnico di Torino, Turin, Italy}
% \cortext[AA]{Corresponding author:lorenzo.crippa@uni-hamburg.de}
% \cortext[BB] {Corresponding author:giuli@sissa.it}
% \cortext[CC]{Corresponding author:bellomia@sissa.it}

\begin{abstract}
  
\end{abstract}

\begin{keyword}
  %% keywords here, in the form: keyword \sep keyword
  Exact Diagonalization \sep
  Quantum Impurity Models\sep  
  Strongly Correlated Electrons \sep  
  Dynamical Mean-Field Theory
\end{keyword}

\end{frontmatter}

% Computer program descriptions should contain the following
% PROGRAM SUMMARY.
\noindent
{\bf PROGRAM SUMMARY}
\begin{small}
  \noindent
  \\
  {\em Program Title:}  EDIpack2                                        \\
{\em Licensing provisions:} GPLv3\\
{\em Programming language:}  Fortran, Python \\
{\em Classification:} 6.5, 7.4, 20 \\
{\em Required dependencies:} CMake ($>=3.0.0$), Scifortran, MPI\\
{\em Nature of problem:}. \\
{\em Solution method:} .\\
\end{small}


\tableofcontents

\subfile{01_intro.tex}
% ##################################################################
% ##################################################################
% ##################################################################


\subfile{02_install.tex}

% ##################################################################
% ##################################################################
% ##################################################################

\subfile{03_edipack2.tex}



%%%%%%%%%%%%%%%%%%%%%%%%%%%%%%%%%%%%%%%%%%%%%%%%%%%%%%%%%%%%%%%%%%
%%%%%%%%%%%%%%%%%%%%%%%%%%%%%%%%%%%%%%%%%%%%%%%%%%%%%%%%%%%%%%%%%%
%%%%%%%%%%%%%%%%%%%%%%%%%%%%%%%%%%%%%%%%%%%%%%%%%%%%%%%%%%%%%%%%%%
%%%%%%%%%%%%%%%%%%%%%%%%%%%%%%%%%%%%%%%%%%%%%%%%%%%%%%%%%%%%%%%%%%

\subfile{04_cbinding.tex}



%%%%%%%%%%%%%%%%%%%%%%%%%%%%%%%%%%%%%%%%%%%%%%%%%%%%%%%%%%%%%%%%%%
%%%%%%%%%%%%%%%%%%%%%%%%%%%%%%%%%%%%%%%%%%%%%%%%%%%%%%%%%%%%%%%%%%
%%%%%%%%%%%%%%%%%%%%%%%%%%%%%%%%%%%%%%%%%%%%%%%%%%%%%%%%%%%%%%%%%%
%%%%%%%%%%%%%%%%%%%%%%%%%%%%%%%%%%%%%%%%%%%%%%%%%%%%%%%%%%%%%%%%%%

\subfile{05_examples.tex}


%%%%%%%%%%%%%%%%%%%%%%%%%%%%%%%%%%%%%%%%%%%%%%%%%%%%%%%%%%%%%%%%%%
%%%%%%%%%%%%%%%%%%%%%%%%%%%%%%%%%%%%%%%%%%%%%%%%%%%%%%%%%%%%%%%%%%
%%%%%%%%%%%%%%%%%%%%%%%%%%%%%%%%%%%%%%%%%%%%%%%%%%%%%%%%%%%%%%%%%%
%%%%%%%%%%%%%%%%%%%%%%%%%%%%%%%%%%%%%%%%%%%%%%%%%%%%%%%%%%%%%%%%%%


\section{Conclusions}
{\color{red} Tentative conclusions}
We have presented EDIpack2, a versatile, high-performance solver for
generic quantum impurity problems. Leveraging on the massively
parallel algorithms introduced with its predecessor, this version of
the library features new capabilities and symmetries within a unified
framework, providing reliable evaluations of the dynamical correlation
functions on arbitrary complex argument.
This makes it particularly well-suited for capturing the sharp
low-energy excitations and complex spectral features of strongly
correlated systems. We have shown that this new version enables the evaluation of impurity
reduced density matrices directly from Fock space quantities and thus unlock 
quantum information analyses of correlated systems.  

A central feature of EDIpack2 is its strong focus on interoperability,
achieved through modern Fortran constructs, C/C++-bindings, and
comprehensive APIs for Python (EDIpy2). 
These interfaces enable seamless integration with broader
computational frameworks like TRIQS and W2Dynamics, expanding the
functionality of these platforms and providing a robust foundation for
reproducible research.

We thoroughly discussed the implementation of the EDIpack2 library, the
most important algorithms and classes. We presented in
details the third-party interfaces which extend exact diagonalization
capabilities beyond the realm of the library itself.   
Finally, we showcased the use of the EDIpack2 software in different
contexts via elaborated example which can serve as reference for the
potential users. 

The modular and extensible design of EDIpack2 provides a
natural foundation for future extensions, for instance to cluster-DMFT where the
impurity problem is generalized to include clusters of interacting
sites. This approach would enable the treatment of spatial
correlations beyond the single-site approximation, capturing effects
such as d-wave pairing in high-temperature superconductors, charge
order, and complex spin textures in strongly correlated systems.
We anticipate that EDIpack2 will become a valuable tool for the
computational condensed matter community, supporting a wide range of
studies on strongly correlated materials and providing a reliable
reference platform for quantum impurity solvers.



%%%%%%%%%%%%%%%%%%%%%%%%%%%%%%%%%%%%%%%%%%%%%%%%%%%%%%%%%%%%%%%%%%
%%%%%%%%%%%%%%%%%%%%%%%%%%%%%%%%%%%%%%%%%%%%%%%%%%%%%%%%%%%%%%%%%%
%%%%%%%%%%%%%%%%%%%%%%%%%%%%%%%%%%%%%%%%%%%%%%%%%%%%%%%%%%%%%%%%%%
%%%%%%%%%%%%%%%%%%%%%%%%%%%%%%%%%%%%%%%%%%%%%%%%%%%%%%%%%%%%%%%%%%

\section*{Acknowledgements}

\section{Appendix A: Monicelli interface}

(scherzo eh)

Tarapia tapioco, prematurata l'interfaccia, o scherziamo? Scusi noi siamo in Monicelli, come fosse un linguaggio esoterico basato su C++ e utilizzante la toolchain di LLVM anche per Linux e Mac Os soltanto in due, oppure in quattro anche scribai con il file sorgente {\tt hm\_bethe.mc}? 

Come {\tt https://github.com/lcrippa/prematurata\_la\_dmft} per esempio. No dico, mi porga il file {\tt bagaglio.cpp}. Le vede le funzioni? Lo vede che interfacciano gli array, non supportati da Monicelli, e prematurano anche?
Ora io le direi, anche con il rispetto per l'autorità, anche solo le due parole come install Monicelli from {\tt https://github.com/esseks/monicelli.git} and run Make which compiles and links the main file {\tt hm\_bethe.mc}, per esempio. Basta così, mi seguano nel programma di test. No no, Attenzione: loop dmft completo supportato secondo la ref. \cite{EDIpack} abbia pazienza. Se no, plotta i dati, anche un pochino Green's function e Self-energy in prefettura. Senza contare che {\tt prematurata\_la\_dmft} ha perso i contatti con il tarapia tapioco, dopo.

Si farà finta di passar da bischeri.

%% References with bibTeX database:
% \section*{References}
\bibliographystyle{elsarticle-num}
\bibliography{references}






\end{document}








