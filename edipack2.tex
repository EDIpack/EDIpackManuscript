%% Use the options 1p,twocolumn; 3p; 3p,twocolumn; 5p; or
%% 5p,twocolumn, preprint, review
\documentclass[final,3p,10pt]{elsarticle}

%%%%%%%%%%%%%%%%%%%%%%%%%%%%%%%%%%%%%%%%%%%%%%%%%%%%%%%

\usepackage[english]{babel}
\usepackage{amsmath,amssymb,amsfonts,amsthm}
\usepackage{makecell}



\usepackage{xcolor}


%% natbib.sty is loaded by default. natbib options with \biboptions{...}
\biboptions{sort&compress,super}

% Hypelinks in the document; settings
\usepackage[colorlinks=true,linkcolor=blue,citecolor=red]{hyperref}
% \usepackage[normalem]{ulem}


\usepackage{listings}
\lstdefinestyle{fstyle}{
  frame=lines,
  language=fortran,
  basicstyle=\footnotesize,
  stringstyle=\ttfamily,
  commentstyle=\itshape,
  fontadjust=true,
  keywordstyle=\color{magenta},
  mathescape,
  numbers=left, numberstyle=\tiny, stepnumber=1, numbersep=3pt
}


\usepackage{array}
\usepackage{tabularx}
\usepackage{ltablex}

\renewcommand{\arraystretch}{1.4}
% \newcolumntype{T}[1]{>{\tt\footnotesize}m{{#1}}}
% \newcolumntype{D}[1]{>{\it\footnotesize}m{#1}}
% \newcolumntype{M}[1]{>{\scriptsize}m{#1}}

\newcolumntype{T}[1]{>{\tt\footnotesize\raggedright\arraybackslash}p{#1}}
\newcolumntype{D}[1]{>{\it\footnotesize\raggedright\arraybackslash}p{#1}}
\newcolumntype{M}[1]{>{\scriptsize\raggedright\arraybackslash}p{#1}}


\newcommand{\onlinecite}[1]{\nocite{#1}\hspace{-0.1cm}\citenum{#1}}

\newcommand {\note}[1]{{\color{blue} [{\bf NOTE}: \bf #1]}}
\newcommand {\aac}[1]{{\color{red} [{\bf AA}: \bf #1]}}
\newcommand {\new}[1]{{\color{blue}\it #1}}
%\newcommand {\new}[1]{{#1}}
% \DeclareMathAlphabet\mathbfcal{OMS}{cmsy}{b}{n}


%Reference to a given labelled equation
%and definition of a bib. element.
%-------------------------------------------
\newcommand{\equ}[1]
{Eq.~(\ref{#1})}

\newcommand{\figu}[1]
{Fig.~\ref{#1}}

\newcommand{\secu}[1]
{Sec.~\ref{#1}}

\newcommand{\ket}[1]
{|#1\rangle}

\newcommand{\bra}[1]
{\langle #1|}

\newcommand{\sgn}
{\mathop{\mathrm{sgn}}}



%SIMBOLI VARI
%%%%%%%%%%%%%%%%%%%%%%%%%%%%%%%%%%%%%%%%%%%%%%%%%%%%%%%
\def\bcen{\begin{center}}
\def\ecen{\end{center}}

\def\a{\alpha}       \def\b{\beta}   \def\g{\gamma}   \def\d{\delta}
\def\e{\varepsilon}  \def\z{\zeta}   \def\h{\eta}     \def\th{\theta}
\def\k{\kappa}       \def\l{\lambda} \def\m{\mu}      \def\n{\nu}
\def\x{\xi}          \def\p{\pi}     \def\r{\rho}     \def\s{\sigma}
\def\t{\tau}         \def\f{\varphi} \def\ph{\varphi} \def\c{\chi}
\def\ps{\pi}        \def\y{\upsilon}\def\o{\omega}   \def\si{\varsigma}
\def\G{\Gamma}       \def\D{\Delta}  \def\Th{\Theta}  \def\L{\Lambda}
\def\X{\Xi}          \def\P{\Pi}     \def\Si{\Sigma}  \def\F{\Phi}
\def\Ps{\Psi}        \def\O{\Omega}  \def\Y{\Upsilon} \def\lg{\langle}

\def\PP{{\cal P}}\def\EE{{\cal E}}\def\MM{{\cal M}} \def\VV{{\cal V}}
\def\CC{{\cal C}}\def\FF{{\cal F}}\def\HH{{\cal H}}\def\WW{{\cal W}}
\def\TT{{\cal T}}\def\NN{{\cal N}}\def\BB{{\cal B}} \def\II{{\cal I}}
\def\RR{{\cal R}}\def\LL{{\cal L}}\def\JJ{{\cal J}} \def\OO{{\cal O}}
\def\DD{{\cal D}}
\def\AA{{\cal A}}
\def\GG{{\cal G}} \def\SS{{\cal S}}
\def\ZZ{{\cal Z}} \def\UU{{\cal U}}
\def\SB{{\cal S}{\cal B}}
\def\aa{{\V \a}}
\def\hh{{\V h}}\def\HHH{{\V H}}
%\def\AA{{\V A}}
%\def\GG{{\V G}}\def\BB{{\V B}}\def\aaa{{\V a}}\def\bbb{{\V b}}
\def\nn{{\V \n}}\def\pp{{\V p}}\def\mm{{\V m}}\def\qq{{\bf q}}
\def\RRR{\mathbb{R}} \def\CCC{\mathbb{C}} \def\NNN{\mathbb{N}}
\def\ZZZ{\mathbb{Z}}
%\def\TTT{\hbox{\msytw T}}



\def\ul{\underline}
\def\=={\equiv}
\def\defi{{\buildrel def \over =}}
\def\lft{\left} \def\rgt{\right} \def\dpr{\partial} \def\der{{\rm d}}
\def\us{\underline \s} \def\ue{{\underline \e}}
\def\la{\left\langle}
\def\ra{\right\rangle}
\def\qed{\raise1pt\hbox{\vrule height5pt width5pt depth0pt}}
\def\iome{i\omega_n} \def\iom{i\omega} \def\iom#1{i\omega_{#1}}
\def\iomn{i\omega_n}
\def\epsk{\epsilon({\bf k})} \def\Ga{\Gamma_{\alpha}}
\def\Seff{S_{eff}}  \def\dinf{$d\rightarrow\infty\,$}
\def\cG0{{\cal G}_0}
\def\cG{{\cal G}}  \def\cU{{\cal U}}  \def\cS{{\cal S}}
\def\spinup{\uparrow} \def\spindown{\downarrow} \def\spindw{\downarrow}
\def\up{\uparrow} \def\down{\downarrow} \def\dw{\downarrow}


\def\Ak{{\bf A}} \def\Akt{{\bf A}(t)} \def\Ek{{\mathbf E}}
% \def\Im{\mbox{Im}}
\def\=={\equiv}
\def\defi{{\buildrel def \over =}} \def\nt{\widetilde{n}}
\def\Im{{\rm Im}} \def\Re{{\rm Re}} \def\Tr{{\rm Tr}\,}
\def\det{{\rm det}\,} 


\def\ibra{\langle}
\def\iket{\rangle}

\def\ka{{\bf k}}
\def\vk{{\bf k}}
\def\qa{{\bf q}}
\def\vQ{{\bf Q}}
\def\vr{{\bf r}}
\def\q{{\bf q}}
\def\R{{\bf R}}
\def\vR{{\bf R}}
\def\kx{{ k_x}}
\def\ia{{\bf i}}
\def\ja{{\bf j}}

\usepackage{bbold}
\def\11{\mathbb{1}}
\def\00{\mathbf{0}}
\def\NAME{{\rm EDIpack2 }}

%%%%%%%%%%%%%%%%%%%%%%%%%%%%%%%%%%%%%%%%%%%%%%%%%%%%%%% 


 
\journal{Computer Physics Communications}

\begin{document}

\begin{frontmatter}

\title{EDIpack2: interoperable Lanczos-based exact diagonalization solver for
  generic quantum impurity problems.}
% \author[]{L.~Crippa}
% \author[]{I.~Krivenko}
% \author[]{S.~Giuli}
% \author[]{G.~Bellomia}
% \author[]{....}
% % \author[]{A.~Scazzola}
% % \author[]{F.~Petocchi}
% % \author[]{G.~Mazza}
% % \author[]{L.~de Medici}
% % \author[]{M.~Capone}
% \author[a]{A.~Amaricci\corref{author}}

% \cortext[author] {Corresponding author.\\\textit{E-mail address:} amaricci@iom.cnr.it}

% \address[a]{CNR-IOM, Istituto Officina dei Materiali,
%     Consiglio Nazionale delle Ricerche, Via Bonomea 265, 34136 Trieste, Italy}
\begin{abstract}
  
\end{abstract}

\begin{keyword}
  %% keywords here, in the form: keyword \sep keyword
  Exact diagonalization \sep
  Quantum Impurity models\sep  
  Strongly correlated electrons \sep  
  Dynamical Mean-Field Theory
\end{keyword}

\end{frontmatter}

% Computer program descriptions should contain the following
% PROGRAM SUMMARY.
\noindent
{\bf PROGRAM SUMMARY}
\begin{small}
  \noindent
  \\
  {\em Program Title:}  EDIpack2                                        \\
{\em Licensing provisions:} GPLv3\\
{\em Programming language:}  Fortran, Python \\
{\em Classification:} 6.5, 7.4, 20 \\
{\em Required dependencies:} CMake ($>=3.0.0$), Scifortran, MPI\\
{\em Nature of problem:}. \\
{\em Solution method:} .\\
\end{small}

\section{Introduction and Motivation}\label{SecIntro}

A few words about the motivations who led us to develop this
software, possible applications and advantages:


* flexibility: can address generic cases including multi-orbital,
superconducting or spin-non-conserving regimes

* zero and low finite temperatures

* direct access to (well approximated) analytic dynamical functions

* direct access to impurity Fock space quantities

bla bla bla







% ##################################################################
% ##################################################################
% ##################################################################



\section{Structure and derived software}\label{SecStruct}
\NAME is a modular library which contains three principal structures.
At the core is the exact diagonalization solver: \NAME.
Next there is a {\tt EDIpack2ineq} which extends application to the case of multiple
inequivalent impurity problems. Finally, there we provide a Fortran-C
interface, which enables development of additional API or
inter-operability with external libraries.  

\begin{itemize}
\item{\bf EDIpack2.}
This constitutes the  building block of the whole software. This part
implement the  with the Lanczos-based solver for generic quantum
impurity systems encoding different symmetries, i.e. quantum number
conservations and apt to solve multi-orbital problems, also in
presence of coupling to local phonons.  
The \NAME solver has a hierarchical and modular structure: different
sections of the library communicate through a shared memory layer. The
top module of the library is {\tt EDIPACK2} which, once loaded,
enables access to the Fortran API in terms of suitable procedures to
initialize, execute and finalize the solver or to retrieve internal
quantities while making opaque to the user the internal structure of
the library. 
A detailed presentation of the library can be found in
Sec.\ref{SecEDIpack}. 


\item{\bf EDIpack2ineq.}
This part of the software, leveraging on the object oriented concepts
available in modern Fortran, aims to extend the \NAME library to the
case of multiple inequivalent and independent impurity problems. This
is particularly useful while using \NAME as a solver for DMFT in
presence of unit cells with inequivalent atoms, for systems with
somehow broken translational symmetry (e.g. heterostructures, large
supercells, etc.).


\item{\bf EDIpack2 C-bindings.}
\NAME includes a single module implementing a Fortran-C interface of the main
library procedures. The module is developed around the implicit {\tt
  ISO\_C\_BINDING} capabilities of the most recent Fortran
distributions, which enable to translate Fortran procedures directly
to C. In order to overcome all the difficulties related to the internal
structure of the library,  we interfaced all and just the
procedures and the variables exposed to the user. 
This module aims to foster interoperability of \NAME  with different
third party softwares as well as to support development of additional
API. 


% \subsection{Dependent software}
% As a concrete example illustrating the interoperability of the
% software we here illustrate the case of EDIpy2, i.e. the Python API for
% \NAME and present an interface layer to Triqs (Toolbox
% for Reasearch In Quantum Systems). 

\item{\bf EDIpy2.}
This is a simple Python module which provides Python API to the
\NAME Fortran library. This interface is built around the Python
support to C-types, which allows to import the dynamic C-binding
library generated upon building \NAME. The module contains a specific
class, whose methods mirrors through  duck-typing all the available
procedures of \NAME as well as it gives access to relevant shared
control variables.




\item{\bf EDIpack2Triqs.}
This is a thin interface layer from \NAME to Triqs,
built around the Python API of the \NAME library. The exact
diagonalization solver is encapsulated in a dedicated class,
containing the necessary methods to initialize and run a single
instance of the solver. The interface also includes a specific class
encompassing the effective discretized bath structure as well as their 
optimization methods.  

  
\end{itemize}







% ##################################################################
% ##################################################################
% ##################################################################

\section{Installation}\label{SecInstall}
The installation of \NAME is available through CMake which ensures
multi-platforms compabitility and dependencies resolution.  
The software builds into two distinct libraries: {\tt
  libedipack.a} and {\tt libedipack\_cbinding.so}.
The former, alongside the generated Fortan modules files,
wraps the \NAME software. The latter enables interoperability through
bindings to the C language.    

\subsection{Dependencies}
Although \NAME is as much self-contained as possible its development
hinges on two external libraries.
\begin{itemize}
\item {\bf SciFortran}: an open-source Fortran library to support
  mathematical and scientific software development. 
\item {\bf MPI} (optional): a distributed memory parallel communication layer with support to modern Fortran compiler.
\end{itemize}
 
SciFortran provides a solid development platform enabling access to
many algorithms and functions, including standard linear algebra
operations and high-performance Lanczos based algorithms. This
greatly reduces code clutter and development time.
The use of distributed memory parallel environment, although optional,
is required to access scalable parallel diagonalization algorithms
which speed up calculations for large dimensional systems. 

\subsection{Build and Install}
\subsubsection{Source}
The software can be installed from source as follows. The source can
be retrieved directly from its GitHub repository, for instance using:
\begin{lstlisting}[language=Bash]
git clone https://github.com/edipack/EDIpack2.0 EDIpack2
\end{lstlisting}
Then, assuming to be in the software directoru, a conventional
out-of-source building is performed using two different compilations
backends.

\begin{itemize}
  \item {\bf GNU Make}\\
This is the default CMake workflow:
\begin{lstlisting}[language=Bash]
mkdir build
cd build
cmake ..
make -j
make install
\end{lstlisting}


\item{\bf Ninja}

An alternative workflow employs the Ninja building backend with
Fortran support. Ninja is generally faster and automatically supports
multi-threaded building:
\begin{lstlisting}[language=Bash]
mkdir build
cd build
cmake -GNinja ..
ninja
ninja install
\end{lstlisting}
\end{itemize}

\noindent
The CMake configurations can be further tuned using the following variables:
\begin{center}
\begin{tabular}{ l|l|l } 
 \hline
  {\bf Option}               & {\bf Scope} & {\bf Value} \\
  \hline
  -D{\bf PREFIX}          & Install directory  & $\sim$/opt/EDIpack2/TAG/PLAT/BRANCH\\
  -D{\bf USE\_MPI}       & MPI support  &  True/{\color{red}False}\\
  -D{\bf WITH\_INEQ}   & Inequivalent impurities support & {\color{red}True}/{False}\\
  -D{\bf VERBOSE}      & Verbose CMake output & {\color{red}True}/{False}\\ 
  -D{\bf BUILD\_TYPE} & Compilation flags & {\color{red}RELEASE}/TESTING/DEBUG/AGGRESSIVE \\
 \hline
\end{tabular}
\end{center}

The default target builds and install either the main library and the C-binding.
However, a specific building for each library is available specifying
the required target. A recap message is printed at the end of the
CMake configuration step. 

\subsubsection{Anaconda}
As an alternative we provide for both Linux and OSx systems
installation through Anaconda packages into a virtual
environment containing Python ($>3.10$).

The Conda package installation procedure reads:
\begin{lstlisting}[language=Bash]
conda create -n edipack
conda activate edipack
conda install -c conda-forge -c edipack edipack2
\end{lstlisting}
\noindent
which installs a bundle of Scifor and \NAME libraries together with
specific {\tt pkg-config} configurations files which can be used to
retrieve compilation and linking flags. 


\subsection{OS Loading}
In oder to avoid possible conflicts or require administrative
privileges, the building step results get installed by default in a user home
directory, specified by the CMake variable {\tt PREFIX}.
In doing so, however, one misses the chance of automatic loading into
the operative system.

We offer different strategies to perform this action:
\begin{enumerate}
\item  A CMake generated configuration file for environment module
  which allows to load and unload the library at any time. This is
  preferred solution for HPC systems. 
\item A CMake generated bash script to be sourced (once or
  permanently) in any shell session to add \NAME library to the
  default environment.
\item A CMake generated pkg-config configuration file to be added in
  the pkg-config path itself.  
\end{enumerate}
An automatically generated recap message with all instructions is
generated at the end of the installation procedure. 



\subsection{Python API}
\subsubsection{Build from source}
EDIpy2, i.e. the Python API of EDIpack2, is available as a stand-alone
module which depends on both \NAME and SciFortran. The package can be
obtained from the repository
\href{https://github.com/EDIpack/EDIpy2.0}{EDIpy2}.

\begin{lstlisting}[language=Bash]
git clone https://github.com/edipack/EDIpy2 EDIpy2
cd EDIpy2
pip install . 
\end{lstlisting}
In some more recent Python distribution the flag {\tt
  --break-system-packages} might be required to complete
installation or a virtual environment should be used instead. 

\subsubsection{Anaconda}
As for EDIpack2, also the Python API in EDIpy2 are available through
Anaconda packaging. In this case the resolution of the dependencies is
taken care from Conda itself:

\begin{lstlisting}[language=Bash]
conda create -n edipack
conda activate edipack
conda install -c conda-forge -c edipack edipack2
\end{lstlisting}




\subsection{TRIQS interface}
A purely Python \NAME to Triqs interface is available, leveraging on
the C-bindings and Python API. The corresponding module depends on
\NAME (which ultimately depends on SciFortran) and Triqs.
Assuming the two software are correctly installed in the OS, the
EDIpack2Triqs interface is installed as follows:

\begin{lstlisting}[language=Bash]
git clone https://github.com/krivenko/edipack2triqs
cd edipack2triqs
pip install .
\end{lstlisting}






% ##################################################################
% ##################################################################
% ##################################################################





\section{Implementation}\label{SecEDIpack}
Here we present an overview of the implementation of the
different parts of the \NAME library. 

\subsection{The quantum impurity problem}\label{sSecQIM}

\subsection{The  basis states}\label{sSecBasis}

\subsection{Conserved quantum numbers}\label{sSecQNs}

\subsection{Krylov based Diagonalization}\label{sSecHam}

\subsection{Dynamical correlation functions}\label{sSecGF}

\subsection{Observables}\label{sSecObc}

\subsection{Reduced impurity density matrix}\label{sSecRDM}

\subsection{Bath parametrization}\label{sSecBath}

\subsection{Bath Optimization}\label{sSecFit}

\subsection{Input/Output}\label{sSecIO}




\section{Inequivalent impurities}

\section{C-bindings}\label{SecCbind}

\section{Python API}

\section{Triqs interface}\label{SecEDI2Triqs}


% ##################################################################
% ##################################################################
% ##################################################################


\section{Usage}\label{SecUsage}
\subsection{Bethe lattice DMFT (Fortran API)}
\subsection{Attractive Hubbard model (Python API)}
\subsection{Multi-orbital Hubbard (Triqs)}









% ##################################################################
% ##################################################################
% ##################################################################





\section{Conclusions}





% ##################################################################
% ##################################################################
% ##################################################################





\section*{Acknowledgements}



%% References with bibTeX database:
% \section*{References}
\bibliographystyle{elsarticle-num}
\bibliography{references}






\end{document}








