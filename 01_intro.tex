\documentclass[edipack2.tex]{subfiles}
\begin{document}

\section{Introduction and Motivation}\label{SecIntro}
{\color{red} A tentative Introduction scheme}


\paragraph{intro} Quantum impurity models play a central role in condensed matter
physics, providing the theoretical foundation for understanding
of strongly correlated electrons systems.
These models describe a small
number of interacting electronic degrees of freedom coupled to 
a non-interacting bath.
Notably, quantum impurity problems lie at the core of dynamical mean-field theory
(DMFT). By  reducing the lattice many-body problem to a
self-consistent quantum impurity problem,  retaining essential local
correlations, the DMFT enabled a  successful effective description of the
complex physics of correlated materials, including Mott insulators,
heavy fermion compounds, and high-temperature superconductors.  
However, solving the quantum impurity problems emerging in the DMFT context accurately and efficiently remains a
critical challenge, particularly for multi-orbital systems, systems
at low temperatures or endowed with lower symmetry. 


\paragraph{what is new, what is inside: modes, RDM, ineq}
  In this work we introduced EDIpack2: a flexible, high-performance numerical library designed to
address these challenges, providing an exact diagonalization solver
for generic quantum impurity problems. 
The library builds on the foundations of
its predecessor featuring massively parallel Lanczos-based algorithms while including several significant
improvements. EDIpack2 supports both zero and finite temperature
calculations for different symmetries of the problem  
within a unified framework, making it suitable for a wide
range of physical phenomena, from spin-orbit coupling effect in
quantum materials to multi-orbital superconductivity. It also provides direct
access to well-approximated dynamical correlation functions across the
entire complex plane, thus enabling calculations of spectral functions, local susceptibilities, and
other response properties.


Additionally, EDIpack2 offers access to the Fock space of
the impurity system and thus, for instance, the possibility to
evaluate the impurity reduced density matrix
(iRDM). This capability enables quantum information-inspired analyses,
providing insights into the entanglement properties, subsystem purity,
and quantum correlations of interacting electron systems. These
features are increasingly recognized as essential for understanding
the emergent behavior in correlated matter, including quantum
criticality and superconductivity.

This new version includes support for systems with inequivalent
impurities, a critical feature for studying heterostructures,
superlattices, and disordered systems, where the local environment
varies significantly across different sites. This capability makes it
possible to explore a wide range of inhomogeneous systems, including
oxide interfaces or disordered systems, where
translational symmetry is broken.


\paragraph{Interoperability}
Interoperability is a core design principle of this version of 
EDIpack. The library is implemented using modern Fortran constructs, with comprehensive
C-bindings that enable integration with other programming
languages. In particular, EDIpack2 includes native  C and C++
interface layer which enable development of third-party
extensions. A significant contribution is realized via the Python API, EDIpy2, which provides a
high-level interface for defining and solving impurity problems. The
Python bindings also enable integration with established scientific
libraries, such as NumPy and SciPy, facilitating data analysis and
rapid prototyping.

The development of such Python API unlocked construction of direct interfaces to the TRIQS
(Toolbox for Research on Interacting Quantum Systems)  as
well as to the W2Dynamics software.
This integration provides EDIpack2 users with access to the
extensive data handling and analysis tools in the TRIQS ecosystem,
while also allowing TRIQS and W2Dynamics users to benefit from the
flexibility and performance of EDIpack2. This dual functionality
extends the capabilities of these platforms, supporting
reproducibility and cross-validation of numerical results, and sets a
new standard for interoperability in quantum impurity solvers.

\paragraph{closure}
In this paper, we present the mathematical framework, implementation
details, and key features of EDIpack2, along with several illustrative
applications. We aim to demonstrate how this flexible,
high-performance solver can be used to tackle a wide range of
challenging impurity problems. 




\end{document}



