\documentclass[edipack2.tex]{subfiles}
\begin{document}

\section{Introduction and Motivation}\label{SecIntro}
\fixme{A tentative Introduction scheme}


Quantum impurity models play a central role in the study of strongly correlated electron systems, capturing the coupling of a small number of localized degrees of freedom with extended non-interacting environment.   
Originally introduced to describe magnetic impurities in metallic hosts and leading to the solution of the Kondo problem, these systems progressively became powerful tools for exploring a wide range of quantum many-body phenomena. More recently, their importance has extended beyond the realm of condensed matter, influencing understanding of other areas such as quantum information or ultra-cold atomic systems. 

At the heart of quantum impurity models is the interaction between a localized ``impurity"  and a surrounding bath of itinerant particles, usually represented as a conduction band of electrons. The definition of the impurity is rather generic, and it can contain lattice (i.e. inequivalent sites), orbital and spin local degrees of freedom.
This setup captures many essential aspects of strong correlation including screening, decoherence and entanglement in a computationally tractable way. 
Most notably, quantum impurity problems lie at the core of dynamical mean-field theory (DMFT) where the complex behavior of the lattice systems is reduced to an effective self-consistent quantum impurity problem.  
By accounting for local quantum fluctuations, DMFT can successfully describe of the most important physical features of correlated materials, including Mott insulators, heavy fermion compounds, and high-temperature superconductors.  

Accurately and efficiently solving the DMFT quantum impurity problem remains a
critical challenge, particularly for multi-orbital systems, systems
at low temperatures or endowed with lower symmetry. During many years of development, different advanced numerical techniques have been developed together with cutting edge approaches aimed to reduce computational load.  

%EDIPACK 2.0
In this work we introduce  \NAME{}: a flexible, high-performance numerical library designed to address these challenges, providing an exact diagonalization (ED) solver for generic Quantum Impurity Problems. 
The library builds on the foundations of
its predecessor, featuring massively parallel Lanczos-based algorithms while including several significant extensions and 
improvements. The library now supports, within a unified framework, both zero and finite temperature calculations for a wide set of impurity models, making it capable of addressing a large set of problems, from spin-orbit coupling effect in quantum materials to multi-orbital superconductivity. It also provides direct
access to well-approximated dynamical correlation functions across the
entire complex plane, thus enabling calculations of spectral functions, local susceptibilities, and other response properties.

Additionally, \NAME offers access to the Fock space of
the impurity system and thus, for instance, the possibility to
evaluate the impurity reduced density matrix
(iRDM). This capability enables quantum information-inspired analyses, providing insights into the entanglement properties, subsystem purity, and quantum correlations of interacting electron systems. These
features are increasingly recognized as essential for understanding the emergent behavior in correlated matter, including quantum criticality and superconductivity.

This new version also includes support for systems with non-equivalent impurities , a critical feature to study hetero-structures, super-lattices, and disordered systems, where the local environment varies significantly across different sites. This capability makes it 
possible to explore a wide range of inhomogeneous systems, including
oxide interfaces or disordered systems, where
translational symmetry is broken.


%\paragraph{Interoperability}\\
With the growing importance of  quantum impurity models in the study of correlated materials,
standardizazion and interoperability between solution software suites has
become a critical necessity.
\NAME has been expressly designed following this guiding principle. 
While the library itself is implemented using modern Fortran constructs, it also provides an extensive set of C-bindings that enable integration with other programming languages,
be they compiled (C, C++), JIT-compiled (Julia) or interpreted (Python). 
Especially notable in this sense is the Python API, EDIpy which, beyond providing a 
high-level interface for defining and solving impurity problems, also allows integration with established scientific libraries, such as NumPy and SciPy, facilitating data analysis and
rapid prototyping.

A key development feature in \NAME, unlocked by the introduction of Python and C/C++ API, 
is the direct  interfaces to the TRIQS (Toolbox for Research on Interacting Quantum Systems) and
W2Dynamics software suites, which are two of the most important and used computational tools in
the field. \NAME seamlessly slots in as an alternative solver to the native Quantum Monte Carlo 
implementations, greatly extending the parameter (and especially temperature) range reachable by 
the simulation while minimizing the cfor the end user.
At the same time, this integration provides \NAME users with access to the
extensive data handling and analysis tools in the TRIQS ecosystem. 
Overall, interoperability extends the cumulative reach and trustworthiness of all these platforms,
enhancing reproducibility and cross-validation of numerical results, with the ultimate aim of achieving standardization for quantum impurity solvers.

%\paragraph{closure}
In this paper, we present the mathematical framework, implementation
details, and key features of \NAME, along with several illustrative
applications. We aim to demonstrate how this flexible,
high-performance solver can be used to tackle a wide range of
challenging impurity problems. 
The rest of this work has the following structure. In \secu{SecInstall} we 
briefly illustrate the overall structure of \NAME together with its dependencies, configuration
and installation procedures. In the following  \secu{SecEDIpack} we present the quantum impurity
problem and review the software implementation in full details. In the following 
\secu{SecInterop} we thoroughly discuss the C-binding interface which is at the heart of the 
interoperability capabilities of \NAME. In this section we also discuss 4 interface layers: i) 
the Python API, ii) the TRIQS interface, iii) the W2Dynamics interface and iv) a basic Julia API.
Finally, in \secu{SecExamples} we discuss the functionalities and the capabilities of \NAME and 
all the discussed extensions with a series of illustrative examples. 


\end{document}



