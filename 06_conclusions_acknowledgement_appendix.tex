\documentclass[edipack2.tex]{subfiles}
\begin{document}




\section{Conclusions}
{\color{red} Tentative conclusions}
We have presented \NAME{}, a versatile, high-performance solver for
generic quantum impurity problems. Leveraging on the massively
parallel algorithms introduced with its predecessor, this version of
the library features new capabilities and symmetries within a unified
framework, providing reliable evaluations of the dynamical correlation
functions on arbitrary complex argument.
This makes it particularly well-suited for capturing the sharp
low-energy excitations and complex spectral features of strongly
correlated systems. We have shown that this new version enables the evaluation of impurity
reduced density matrices directly from Fock space quantities and thus unlock 
quantum information analyses of correlated systems.  

A central feature of \NAME is its strong focus on interoperability,
achieved through modern Fortran constructs, C/C++-bindings, and
comprehensive APIs for Python (EDIpy2). 
These interfaces enable seamless integration with broader
computational frameworks like TRIQS and W2Dynamics, expanding the
functionality of these platforms and providing a robust foundation for
reproducible research.

We thoroughly discussed the implementation of the \NAME library, the
most important algorithms and classes. We presented in
details the third-party interfaces which extend exact diagonalization
capabilities beyond the realm of the library itself.   
Finally, we showcased the use of the \NAME software in different
contexts via elaborated example which can serve as reference for the
potential users. 

The modular and extensible design of \NAME provides a
natural foundation for future extensions, for instance to cluster-DMFT where the
impurity problem is generalized to include clusters of interacting
sites. This approach would enable the treatment of spatial
correlations beyond the single-site approximation, capturing effects
such as d-wave pairing in high-temperature superconductors, charge
order, and complex spin textures in strongly correlated systems.
We anticipate that \NAME will become a valuable tool for the
computational condensed matter community, supporting a wide range of
studies on strongly correlated materials and providing a reliable
reference platform for quantum impurity solvers.



%%%%%%%%%%%%%%%%%%%%%%%%%%%%%%%%%%%%%%%%%%%%%%%%%%%%%%%%%%%%%%%%%%
%%%%%%%%%%%%%%%%%%%%%%%%%%%%%%%%%%%%%%%%%%%%%%%%%%%%%%%%%%%%%%%%%%
%%%%%%%%%%%%%%%%%%%%%%%%%%%%%%%%%%%%%%%%%%%%%%%%%%%%%%%%%%%%%%%%%%
%%%%%%%%%%%%%%%%%%%%%%%%%%%%%%%%%%%%%%%%%%%%%%%%%%%%%%%%%%%%%%%%%%

\section*{Acknowledgements}

\section{Appendix A: Monicelli interface}\label{appendixA}
In order to further showcase the interoperability capabilities of
\NAME provided by the C-bindings module, we describe a simple interface to {\tt Monicelli}, see
\href{github.com/esseks/monicelli.git}{ttps://github.com/esseks/monicelli.git}. 

  
{\tt Monicelli} is an esoteric programming language based on the LLVM
toolchain.
It is written in C++ and offers wrappers for the basic
types and operations thereof. The syntax replicates a cultural italian
phenomenon known as ``supercazzola", a rambling nonsensical discourse
which gives the false impression of carrying an actual meaning,
introduced in the trilogy movies ``Amici Miei'', directed by Mario
Monicelli (see
\href{https://it.wikipedia.org/wiki/Amici_miei}{it.wikipedia.org/wiki/Amici\_miei}). 

The {\tt Monicelli} language can be found at
\href{https://github.com/esseks/monicelli.git}{github.com/esseks/monicelli.git}
and can be installed using CMake. The software depends on C++
compiler with {\tt stdlib}, LLVM and in some cases {\tt libz} library. The installation gives access to the {\tt
  Monicelli} compiler {\tt mcc}, which statically links LLVM. 
We believe the discourse flow of this language is best captured by 
illustrating its features via a paraphrase of the original supercazzola, an epic dialogue between a traffic policeman (V), count Mascetti (M) and G. Perozzi (P) (see
\href{https://it.wikipedia.org/wiki/Supercazzola#Origine}{it.wikipedia.org/wiki/Supercazzola}
and \href{https://www.youtube.com/watch?v=SF8YUFdP6eU}{Movie Scene}). 

\noindent
V: {\it Lei ha clacsonato?}\\
P: {\it Tu ha clacsonato?}\\
M: {\it Tarapia tapioco, prematurata l'interfaccia, o scherziamo?}\\
V: {\it Prego}?\\
M: {\it Scusi noi siamo in Monicelli, come fosse un linguaggio esoterico
basato su C++ e utilizzante la toolchain di LLVM anche per Linux e
Unix soltanto in due, oppure in quattro anche scribai con il file
sorgente {\tt hm\_bethe.mc}?  Come {\tt
  github.com/lcrippa/prematurata\_la\_dmft}, per esempio?}\\
V: {\it Ma che DMFT, mi faccia il piacere! Questi signori stavano
programmano loro, non si intrometta!}\\
M: {\it Ma no, dico, mi porga il file {\tt bagaglio.cpp}. Le vede le
funzioni? Lo vede che interfacciano gli array, non supportati da
Monicelli, e prematurano anche! 
Ora io le direi, anche con il rispetto per l'autorità, anche solo le
due parole come install {\tt Monicelli} from
\href{github.com/esseks/monicelli.git}{https://github.com/esseks/monicelli.git}
e compila il file {\tt hm\_bethe.mc}, per esempio.}\\
V: {\it Basta così! Mi seguano nel programma di test!}\\
P: {\it No, no, no, attenzione! Il loop dmft completo \`e supportato
secondo la Ref.~\onlinecite{Georges1996RMP}, abbia pazienza,
senn\`o.... Plotta i dati, anche un pochino di Green's function e
Self-energy in prefettura.}\\
M: {\it Senza contare che {\tt prematurata\_la\_dmft} ha perso i contatti con il tarapia tapioco, dopo.}\\
V: {\it Ho bello che capito. Si farà finta di passar da bischeri!}\\

We shall now present a fully functioning DMFT code for
the solution of the Bethe lattice written in Monicelli. It requires a minimal C++
interface to handle arrays, complex numbers and properly interfacing
the EDIpack functions.
The first part of the DMFT code load all the required functions from
{\tt bagaglio.cpp}, in particular it loads the interface to the \NAME
solver functions {\tt ed\_init\_solver} and {\tt ed\_solve}:

\begin{lstlisting}[style=cstyle]
bituma le funzioni ausiliarie che vengono dal bagaglio
...
blinda la supercazzola leggi o scherziamo?
blinda la supercazzola iniziailsolver o scherziamo?
blinda la supercazzola risolvi o scherziamo?
blinda la supercazzola prendilasigma o scherziamo?
blinda la supercazzola prendilbagno con l`elemento Necchi, il valore Sassaroli o scherziamo?
...
\end{lstlisting}

After a long initialization, we follow the structure already presented
in \secu{SecExamplesBetheDMFT} for the Fortran implementation.
The DMFT code starts with the opening {\tt Lei ha clacsonato} (see
above). Next, we read the input file and set the dimensions of some
array describing the bath, the self-energy and Green's
functions. We start a DMFT iteration loop using the command {\tt
  stuzzica}, which includes an internal frequency loop as {\tt
  Monicelli} does not support array algebra.
The loop contains a call to the \NAME solver function, then proceeds
by retrieving the self-energy function which is used to obtain the
local Green's function. These enters the self-consistency equation
which update the Weiss field. Finally bath optimization is performed
and if error condition is met the loop exit.
The main step of the implementation reads: 


\begin{lstlisting}[style=cstyle]
Lei ha clacsonato
#Read the input file
    prematurata la supercazzola dimmilfile o scherziamo?
    prematurata la supercazzola leggi o scherziamo? 
...                                                  
#Init the ED solver:
    voglio il sapone, Necchi come se fosse
    prematurata la supercazzola ilbagnoepronto o scherziamo?
    prematurata la supercazzola lavati con il sapone o scherziamo?
        
    prematurata la supercazzola prendilah o scherziamo?
    prematurata la supercazzola iniziailsolver o scherziamo?
...
#DMFT loop:
stuzzica    
      prematurata la supercazzola risolvi o scherziamo?
      
      prematurata la supercazzola prendilag o scherziamo?
      prematurata la supercazzola prendilasigma o scherziamo?
#Get $G_{loc}$ and update $\GG^{-1}$: frequency loop
      stuzzica                   
          il contatoredue come fosse 0
          prematurata la supercazzola prendi con 108, contatore, 0.0, 0.0, 0 o scherziamo?
...
      e brematura anche, se il contatore minore di frequenze
...      
      prematurata la supercazzola lavatiancora o scherziamo?
#Fit the bath:       
      prematurata la supercazzola spiaccica o scherziamo?
#Mix the bath           
      prematurata la supercazzola failmischiotto con il sapone, il frullatore o scherziamo?
#Check error      
      la fine come se fosse prematurata la supercazzola cisiamo o scherziamo?         
 e brematura anche, se la fine minore di 1
\end{lstlisting}

\paragraph{Usage}.
The source code can be retrieved from
\href{https://github.com/lcrippa/prematurata_la_dmft}{github.com/lcrippa/prematurata\_la\_dmft}. 
The {\tt src} directory contains several files, including the source
code {\tt hm\_bethe.mc} implementing the DMFT algorithm, an auxiliary file {\tt bagaglio.cpp} which
contains a number of functions implementing complex algebra, array
costruction and interfacing the \NAME procedures. The directory also
contains an example of input file and a converged bath parametrization
for the Bethe lattice solution at $U=2$.

The code is compiled using standard Make invocation in the source
directory:
\begin{lstlisting}[style=mybash]
git clone https://github.com/lcrippa/prematurata_la_dmft
cd prematurata_la_dmft/src
make
\end{lstlisting}
A simple run using provided input file {\tt inputED.conf} will
re-converge the solution within few loops. 



\end{document}
