\documentclass[dvipsnames]{SciPost}
\usepackage[bitstream-charter]{mathdesign}
\usepackage[english]{babel}
\usepackage[T1]{fontenc}
\usepackage{lmodern}
% \usepackage{amsmath}
% \usepackage{amssymb}
%\usepackage{amsfonts}
\usepackage{amsthm}
\usepackage{microtype}
\usepackage{ragged2e}
\usepackage{makecell}
\usepackage{array}
\usepackage{stackengine}
\newlength\llength
\llength=1.38ex\relax
\usepackage{makecell}
\usepackage{subfiles}
\usepackage{comment}
\usepackage{xspace}
\usepackage{array}
\usepackage{tabularx}
\usepackage{ltablex}
%\usepackage[inline]{showlabels} %inline: draft//final
\usepackage{ulem} % \uline,  \uwave, \sout...
\usepackage{xkcdcolors}
\usepackage{dialogue}
\usepackage{lips}
% Prevent all line breaks in inline equations.
\binoppenalty=10000
\relpenalty=10000

  \hypersetup{
    breaklinks=true,
    colorlinks,
    pdfborder={0 0 0},
    linkcolor={red!50!black},
    citecolor={blue!50!black},
    urlcolor={blue!80!black}
  }

\urlstyle{same}

% Fix \cal and \mathcal characters look (so it's not the same as \mathscr)
\DeclareSymbolFont{usualmathcal}{OMS}{cmsy}{m}{n}
\DeclareSymbolFontAlphabet{\mathcal}{usualmathcal}
\DeclareMathAlphabet\mathbfcal{OMS}{cmsy}{b}{n}


\renewcommand{\texttt}[1]{%
  \begingroup
  \ttfamily
  \begingroup\lccode`~=`_\lowercase{\endgroup\def~}{/\discretionary{}{}{}}%
  \begingroup\lccode`~=`/\lowercase{\endgroup\def~}{/\discretionary{}{}{}}%
  \begingroup\lccode`~=`[\lowercase{\endgroup\def~}{[\discretionary{}{}{}}%
  \begingroup\lccode`~=`.\lowercase{\endgroup\def~}{.\discretionary{}{}{}}%
  \catcode`/=\active\catcode`[=\active\catcode`.=\active
  \scantokens{#1\noexpand}%
  \endgroup
}

% \setcounter{secnumdepth}{4} 
% \setcounter{tocdepth}{4}


% Trying to define a scipost-vibe colorset for listings
\definecolor{comment-color}{rgb}{0.8,0.1,0.1}
\definecolor{keyword-color}{rgb}{0.3,0.3,1}
\definecolor{string-color}{rgb}{0.58, 0, 0.82}



%LISTING PACKAGE AND CONFIG:
%%%%%%%%%%%%%%%%%%%%%%%%%%%%%%%%%%%%%%%%
\usepackage{listings}

% backgroundcolor - colour for the background. External color or xcolor package needed.
% commentstyle - style of comments in source language.
% basicstyle - font size/family/etc. for source (e.g. basicstyle=\ttfamily\small)
% keywordstyle - style of keywords in source language (e.g. keywordstyle=\color{red})
% numberstyle - style used for line-numbers
% numbersep - distance of line-numbers from the code
% stringstyle - style of strings in source language
% showspaces - emphasize spaces in code (true/false)
% showstringspaces - emphasize spaces in strings (true/false)
% showtabs - emphasize tabulators in code (true/false)
% numbers - position of line numbers (left/right/none, i.e. no line numbers)
% prebreak - displaying mark on the end of breaking line (e.g. prebreak=\raisebox{0ex}[0ex][0ex]{\ensuremath{\hookleftarrow}})
% captionpos - position of caption (t/b)
% frame - showing frame outside code (none/leftline/topline/bottomline/lines/single/shadowbox)
% breakatwhitespace - sets if automatic breaks should only happen at whitespaces
% breaklines - automatic line-breaking
% keepspaces - keep spaces in the code, useful for indetation
% tabsize - default tabsize
% escapeinside - specify characters to escape from source code to LATEX (e.g. escapeinside={\%*}{*)})
% rulecolor - Specify the colour of the frame-box

% FORTRAN
%%%%%%%%%%%%%%%%%%%%%%%%%%%%%%%%%%%%%%%%
%see fonts here: https://latexref.xyz/Low_002dlevel-font-commands.html
%set with  basicstyle={\usefont{OT1}{lmr}{ec}{n}\scriptsize},  
\lstdefinestyle{fstyle}{
  language={[03]Fortran},
  basicstyle=\ttfamily\linespread{1.15}\footnotesize,
  stringstyle=\color{string-color},
  keywordstyle=\color{keyword-color}\footnotesize,
  commentstyle=\color{comment-color}\itshape\linespread{2}\footnotesize,
  morecomment=[l]{!\ },
   fontadjust=true,
  mathescape,
   breakatwhitespace=false,
   keepspaces=true,
  showstringspaces=false,
  columns=fullflexible,
  frame=none,
  xleftmargin=1pt,
  xrightmargin=1pt,
  backgroundcolor=\color{xkcdOffWhite},
  numbers=left, numberstyle=\tiny, stepnumber=1, numbersep=3pt
}


% C/C++
%%%%%%%%%%%%%%%%%%%%%%%%%%%%%%%%%%%%%%%%
\lstdefinestyle{cstyle}{
  language=C,
  basicstyle=\ttfamily\footnotesize,
  stringstyle=\color{string-color},
  keywordstyle=\color{keyword-color}\footnotesize,
  commentstyle=\color{comment-color}\itshape\footnotesize,
  captionpos=b,
  mathescape,
  fontadjust=true,
  breakatwhitespace=false,
  keepspaces=true,
  showstringspaces=false,
  columns=fullflexible,
  xleftmargin=3.4pt,
  xrightmargin=3.4pt,
  backgroundcolor=\color{xkcdOffWhite},
  frame=none
}


% PYTHON
%%%%%%%%%%%%%%%%%%%%%%%%%%%%%%%%%%%%%%%%
\lstdefinestyle{mypython}{
  language=python,
  basicstyle=\ttfamily\footnotesize,
  stringstyle=\color{string-color},
  keywordstyle=\color{keyword-color},
  commentstyle=\color{comment-color}\itshape\footnotesize,
  fontadjust=true,
  morekeywords={np,plt,oplot,solve,EDIpackSolver,c,c_dag,n,ed,MPI,comm},
  morecomment=[l]{\#\ },
  mathescape,
  fontadjust=true,
  mathescape,
  breakatwhitespace=false,
  keepspaces=true,
  showstringspaces=false,
  columns=fullflexible,
  % frame=single,
  % xleftmargin=3.4pt,
  % xrightmargin=3.4pt,
  backgroundcolor=\color{xkcdOffWhite},
  numbers=none, numberstyle=\tiny, stepnumber=1, numbersep=3pt
}

% BASH
%%%%%%%%%%%%%%%%%%%%%%%%%%%%%%%%%%%%%%%%
\lstdefinestyle{mybash}{
  language=bash,
  basicstyle=\ttfamily\footnotesize,
  stringstyle=\ttfamily,
  keywordstyle=\color{keyword-color},
  commentstyle=\color{comment-color}\itshape\footnotesize,
  morekeywords={*,git,mkdir,cmake,make,pip,ninja,conda},
  alsoletter=-
  deletekeywords={conda-forge},
  captionpos=b,
  mathescape,
  fontadjust=true,
  breakatwhitespace=false,
  keepspaces=true,
  showstringspaces=false,
  columns=flexible,
  xleftmargin=3.4pt,
  xrightmargin=3.4pt,
  backgroundcolor=\color{xkcdOffWhite},
  frame=none
}

%%
%% Julia definition (c) 2014 Jubobs
%%
\lstdefinelanguage{Julia}%
  {morekeywords={abstract,break,case,catch,const,continue,do,else,elseif,%
      end,export,false,for,function,immutable,import,importall,if,in,%
      macro,module,otherwise,quote,return,switch,true,try,type,typealias,%
      using,while},%
   sensitive=true,%
   alsoother={\$},%
   morecomment=[l]\#,%
   morecomment=[n]{\#=}{=\#},%
   morestring=[s]{"}{"},%
   morestring=[m]{'}{'},%
}[keywords,comments,strings]%


\lstdefinestyle{myjulia}{
  frame=lines,
  language=julia,
  basicstyle=\ttfamily\footnotesize,
  stringstyle=\ttfamily,
  commentstyle=\itshape,
  fontadjust=true,
  keywordstyle=\color{keyword-color},
  commentstyle=\color{comment-color}\footnotesize,
  morecomment=[l]{\#\ },% Otherwise it's a comment only with space after #
  mathescape,
  fontadjust=true,
  mathescape,
  % breakatwhitespace=false,
  keepspaces=true,
  showstringspaces=false,
  columns=fullflexible,
  xleftmargin=3.4pt,
  xrightmargin=3.4pt,
  backgroundcolor=\color{xkcdOffWhite},
  frame=none
}



%%
%% Monicelli definition (c) 2025 Amaricci
%%
\lstdefinelanguage{Monicelli}%
  {
  morekeywords={vaffanzum,Lei,ha, clacsonato,più,meno,per,diviso,maggiore,minore,scappellamento,come fosse,Necchi,Mascetti,Perozzi,Melandri,Sassaroli,voglio,posterdati,mi,porga,stuzzica,brematura, anche,che, cos'è, o, magari,tarapia, genio,blinda,prematurata, la, supercazzola, avvertite,con,il,scherziamo,?,bituma},%
   sensitive=true,%
   alsoother={\$},%
   morecomment=[l]\#,%
   morecomment=[n]{\#=}{=\#},%
   morestring=[s]{"}{"},%
   morestring=[m]{'}{'},%
}[keywords,comments,strings]%

\lstdefinestyle{MonicelliStyle}{
  language=Monicelli,
  basicstyle=\ttfamily\footnotesize,
  stringstyle=\ttfamily,
  keywordstyle=\color{keyword-color},
  commentstyle=\color{comment-color}\itshape\footnotesize,
  morecomment=[l]{\#\ },
  mathescape,
  fontadjust=true,
  mathescape,
  breakatwhitespace=false,
  keepspaces=true,
  showstringspaces=false,
  columns=fullflexible,
  xleftmargin=3.4pt,
  xrightmargin=3.4pt,
  backgroundcolor=\color{xkcdOffWhite},
  frame=none
}















\renewcommand{\arraystretch}{1.4}
\newcolumntype{T}[1]{>{\tt\footnotesize\raggedright\arraybackslash}p{#1}}
\newcolumntype{D}[1]{>{\it\footnotesize\raggedright\arraybackslash}p{#1}}
\newcolumntype{M}[1]{>{\scriptsize\raggedright\arraybackslash}p{#1}}


\newcommand{\onlinecite}[1]{\cite{#1}}
% {\nocite{#1}\!\!\!\citenum{#1}} This was for CPC
\newcommand{\fixme}[1]{{\color{BrickRed} #1}}
\newcommand {\note}[1]{{\color{blue} [{\bf NOTE}: \bf #1]}}
\newcommand {\new}[1]{{\color{blue}\it #1}}




%Reference to a given labelled equation
%and definition of a bib. element.
%-------------------------------------------
\newcommand{\equ}[1]
{Eq.\,(\ref{#1})}

\newcommand{\figu}[1]
{Fig.\,\ref{#1}}

\newcommand{\secu}[1]
{Sec.\,\ref{#1}}

\newcommand{\ket}[1]
{|#1\rangle}

\newcommand{\bra}[1]
{\langle #1|}

\newcommand{\sgn}
{\mathop{\mathrm{sgn}}}



%SIMBOLI VARI
%%%%%%%%%%%%%%%%%%%%%%%%%%%%%%%%%%%%%%%%%%%%%%%%%%%%%%%
\def\a{\alpha}       \def\b{\beta}   \def\g{\gamma}   \def\d{\delta}
\def\e{\varepsilon}  \def\z{\zeta}   \def\h{\eta}     \def\th{\theta}
\def\k{\kappa}       \def\l{\lambda} \def\m{\mu}      \def\n{\nu}
\def\x{\xi}          \def\p{\pi}     \def\r{\rho}     \def\s{\sigma}
\def\t{\tau}         \def\f{\varphi} \def\ph{\varphi} \def\c{\chi}
\def\ps{\pi}        \def\y{\upsilon}\def\o{\omega}   \def\si{\varsigma}
\def\G{\Gamma}       \def\D{\Delta}  \def\Th{\Theta}  \def\L{\Lambda}
\def\X{\Xi}          \def\P{\Pi}     \def\Si{\Sigma}  \def\F{\Phi}
\def\Ps{\Psi}        \def\O{\Omega}  \def\Y{\Upsilon} \def\lg{\langle}

\def\PP{{\cal P}}\def\EE{{\cal E}}\def\MM{{\cal M}} \def\VV{{\cal V}}
\def\CC{{\cal C}}\def\FF{{\cal F}}\def\HH{{\cal H}}\def\WW{{\cal W}}
\def\TT{{\cal T}}\def\NN{{\cal N}}\def\BB{{\cal B}} \def\II{{\cal I}}
\def\RR{{\cal R}}\def\LL{{\cal L}}\def\JJ{{\cal J}} \def\OO{{\cal O}}
\def\DD{{\cal D}}
\def\AA{{\cal A}}
\def\GG{{\cal G}} \def\SS{{\cal S}}
\def\ZZ{{\cal Z}} \def\UU{{\cal U}}
\def\SB{{\cal S}{\cal B}}
\def\aa{{\V \a}}
\def\hh{{\V h}}\def\HHH{{\V H}}
%\def\AA{{\V A}}
%\def\GG{{\V G}}\def\BB{{\V B}}\def\aaa{{\V a}}\def\bbb{{\V b}}
\def\nn{{\V \n}}\def\pp{{\V p}}\def\mm{{\V m}}\def\qq{{\bf q}}
\def\RRR{\mathbb{R}} \def\CCC{\mathbb{C}} \def\NNN{\mathbb{N}}
\def\ZZZ{\mathbb{Z}}
%\def\TTT{\hbox{\msytw T}}



\def\ul{\underline}
\def\=={\equiv}
\def\defi{{\buildrel def \over =}}
\def\lft{\left} \def\rgt{\right} \def\dpr{\partial} \def\der{{\rm d}}
\def\us{\underline \s} \def\ue{{\underline \e}}
\def\la{\left\langle}
\def\ra{\right\rangle}
\def\qed{\raise1pt\hbox{\vrule height5pt width5pt depth0pt}}
\def\iome{i\omega_n} \def\iom{i\omega} \def\iom#1{i\omega_{#1}}
\def\iomn{i\omega_n}
\def\epsk{\epsilon({\bf k})} \def\Ga{\Gamma_{\alpha}}
\def\Seff{S_{eff}}  \def\dinf{$d\rightarrow\infty\,$}
\def\cG0{{\cal G}_0}
\def\cG{{\cal G}}  \def\cU{{\cal U}}  \def\cS{{\cal S}}
\def\spinup{\uparrow} \def\spindown{\downarrow} \def\spindw{\downarrow}
\def\up{\uparrow} \def\down{\downarrow} \def\dw{\downarrow}


\def\Ak{{\bf A}} \def\Akt{{\bf A}(t)} \def\Ek{{\mathbf E}}
% \def\Im{\mbox{Im}}
\def\=={\equiv}
\def\defi{{\buildrel def \over =}} \def\nt{\widetilde{n}}
\def\Im{{\rm Im}} \def\Re{{\rm Re}} \def\Tr{{\rm Tr}}
\def\det{{\rm det}\,} 


\def\ibra{\langle}
\def\iket{\rangle}

\def\ka{{\bf k}}
\def\vk{{\bf k}}
\def\qa{{\bf q}}
\def\vQ{{\bf Q}}
\def\vr{{\bf r}}
\def\q{{\bf q}}
\def\R{{\bf R}}
\def\vR{{\bf R}}
\def\kx{{ k_x}}
\def\ia{{\bf i}}
\def\ja{{\bf j}}

%\usepackage{bbold}
\def\11{\mathbb{1}}
\def\00{\mathbf{0}}
\def\NAME{{\rm EDIpack}\xspace}
\newcommand{\shortrightarrow}{\mathrel{\mkern3mu\rightarrow\mkern}}





%%%%%%%%%%%%%%%%%%%%%%%%%%%%%%%%%%%%%%%%%%%%%%%%%%%%%%% 
\fancypagestyle{SPstyle}{
\fancyhf{}
\lhead{\colorbox{scipostblue}{\bf \color{white} ~SciPost Physics Codebases }}
\rhead{{\bf \color{scipostdeepblue} ~Submission }}
\renewcommand{\headrulewidth}{1pt}
\fancyfoot[C]{\textbf{\thepage}}
}



\begin{document}

\pagestyle{SPstyle}

%TITLE
%%%%%%%%%% TODO: Write your article's title here
\begin{center}{
\Large \textbf{
\color{scipostdeepblue}{
A flexible and interoperable high-performance Lanczos-based solver for generic quantum impurity problems: upgrading EDIpack\\
}
}
}
\end{center}
%%%%%%%%%% END TODO: TITLE



%%%%%%%%%% AUTHORS
% Mark the corresponding author(s) with a superscript symbol in this order
% \star, \dagger, \ddagger, \circ, \S, \P, \parallel, ...
\begin{center}\textbf{\small
    Lorenzo Crippa\textsuperscript{1,2,3},
    Igor Krivenko\textsuperscript{1},
    Samuele Giuli\textsuperscript{4},
    Gabriele Bellomia\textsuperscript{4},
    Alexander Kowalski\textsuperscript{3},    
    Francesco Petocchi\textsuperscript{5},
    Alberto Scazzola\textsuperscript{6},
    Markus Wallerberger\textsuperscript{7},
    Giacomo Mazza\textsuperscript{8},
    Luca de Medici\textsuperscript{9},
    Giorgio Sangiovanni\textsuperscript{2,3},
    Massimo Capone\textsuperscript{4,10} and 
    Adriano Amaricci\textsuperscript{10}    
}\end{center}
%%%%%%%%%% END TODO: AUTHORS



%%%%%%%%%% TODO: AFFILIATIONS
% Write all affiliations here.
% Format: institute, city, country
\begin{center}
%   \newcommand{\CNRIOM}{CNR-IOM, Istituto Officina dei Materiali,
%   Consiglio Nazionale delle Ricerche, Via Bonomea 265, 34136
%   Trieste, Italy}
% \newcommand{\SISSA}{Scuola Internazionale Superiore di Studi Avanzati (SISSA),
%   Via Bonomea 265, 34136 Trieste, Italy}
% \newcommand{\ITPHamburg}{I. Institut f\"ur Theoretische Physik,
%   University of Hamburg, Notkestra\ss e 9, 22607 Hamburg, Germany}
% \newcommand{\WZBURG}{Institut f\"ur Theoretische Physik und
%   Astrophysik,Universit\"at W\"urzburg, 97074 W\"urzburg, Germany}
% \newcommand{\CTQMAT}{W\"urzburg-Dresden Cluster of Excellence ct.qmat, 01062 Dresden, Germany}
% \newcommand{\Geneve}{Department of Quantum Matter Physics, University of
%   Geneva, Quai Ernest-Ansermet 24, 1211 Geneva, Switzerland}
% \newcommand{\UPISA}{Department of Physics ``E. Fermi'' University of
%   Pisa, Largo B. Pontecorvo 3, 56127 Pisa, Italy}
% \newcommand{\ESPCI}{LPEM, ESPCI Paris, PSL Research University, CNRS, Sorbonne Universit\'e, 75005 Paris, France}
% \newcommand{\TUW}{Institute of Solid State Physics, TU Wien, 1040 Vienna, Austria}
% \newcommand{\ToPoli}{Department of Electronics and Telecommunications, Politecnico di Torino, I\-10129 Torino, Italy}
%
%No ZIPcode to Brooklyn...
%
  \newcommand{\CNRIOM}{CNR-IOM, Istituto Officina dei Materiali,
  Consiglio Nazionale delle Ricerche, Trieste, Italy}
\newcommand{\SISSA}{SISSA, Scuola Internazionale Superiore di Studi Avanzati, Trieste, Italy}
\newcommand{\ITPHamburg}{I. Institut f\"ur Theoretische Physik,
  University of Hamburg, Hamburg, Germany}
\newcommand{\WZBURG}{Institut f\"ur Theoretische Physik und
  Astrophysik, Universit\"at W\"urzburg, W\"urzburg, Germany}
\newcommand{\CTQMAT}{W\"urzburg-Dresden Cluster of Excellence ct.qmat, Dresden, Germany}
\newcommand{\Geneve}{Department of Quantum Matter Physics, University of
  Geneva, Geneva, Switzerland}
\newcommand{\UPISA}{Department of Physics ``E. Fermi'', University of
  Pisa, Pisa, Italy}
\newcommand{\ESPCI}{LPEM, ESPCI Paris, PSL Research University, CNRS, Sorbonne Universit\'e, Paris, France}
\newcommand{\TUW}{Institute of Solid State Physics, TU Wien, Vienna, Austria}
\newcommand{\ToPoli}{Department of Electronics and Telecommunications, Politecnico di Torino, Torino, Italy}
{\small
{\bf 1} \ITPHamburg\\
{\bf 2} \CTQMAT\\
{\bf 3} \WZBURG\\
{\bf 4} \SISSA\\
{\bf 5} \Geneve\\
{\bf 6} \ToPoli\\
{\bf 7} \TUW\\   
{\bf 8} \UPISA\\
{\bf 9} \ESPCI\\
{\bf 10} \CNRIOM\\
}
\end{center}
%%%%%%%%%% END TODO: AFFILIATIONS



%%%%%%%%%% TODO: EMAIL
% Provide email address of corresponding author(s)
\begin{center}
%\vspace{\baselineskip}
% $\star$ \href{mailto:email1}{\small email1}\,,\quad
EDIpack group: \href{mailto:edipack@cnr.iom.it}{\small edipack@cnr.iom.it}
\end{center}
%%%%%%%%%% END TODO: EMAIL



%%%%%%%%%% TODO: ABSTRACT
% Write your abstract here.
\section*{\color{scipostdeepblue}{Abstract}}
\textbf{\boldmath{%
EDIpack is a flexible, high-performance numerical library to solve generic quantum impurity problems using Lanczos-based exact diagonalization, as emerging from Dynamical Mean-Field Theory description of strongly correlated materials. 
The library efficiently solve impurity problems with different symmetry properties, including superconductivity, spin-orbit terms or electron-phonon coupling. It provides access to dynamical correlation function on the entire complex frequency plane at zero and low-temperatures.    
% We discuss the mathematical framework alongside implementation details of the software. Moreover we showcase illustrative applications.
The modular architecture of the software not only provides Fortran APIs but also includes bindings to C/C++, interfaces with Python and Julia or with TRIQS and W2Dynamics research platforms, thus ensuring unprecedented level of inter-operability.   
The outlook includes further extensions for quantum materials, condensed matter physics, and quantum information applications.
}}
%%%%%%%%%% END TODO: ABSTRACT

\vspace{\baselineskip}


% %%%%%%%%%% BLOCK: Copyright information
% % This block will be filled during the proof stage, and finilized just before publication.
% % It exists here only as a placeholder, and should not be modified by authors.
% \noindent\textcolor{white!90!black}{%
% \fbox{\parbox{0.975\linewidth}{%
% \textcolor{white!40!black}{\begin{tabular}{lr}%
%   \begin{minipage}{0.6\textwidth}%
%     {\small Copyright attribution to authors. \newline
%     This work is a submission to SciPost Physics Codebases. \newline
%     License information to appear upon publication. \newline
%     Publication information to appear upon publication.}
%   \end{minipage} & \begin{minipage}{0.4\textwidth}
%     {\small Received Date \newline Accepted Date \newline Published Date}%
%   \end{minipage}
% \end{tabular}}
% }}
% }
% %%%%%%%%%% BLOCK: Copyright information


%%%%%%%%%% TODO: LINENO
% For convenience during refereeing we turn on line numbers:
\linenumbers
% You should run LaTeX twice in order for the line numbers to appear.
%%%%%%%%%% END TODO: LINENO

%%%%%%%%%% TODO: TOC 
% Guideline: if your paper is longer that 6 pages, include a TOC
% To remove the TOC, simply cut the following block
\vspace{10pt}
\noindent\rule{\textwidth}{1pt}
\tableofcontents
\noindent\rule{\textwidth}{1pt}
\vspace{10pt}
%%%%%%%%%% END TODO: TOC





% ###################################################
% ###################################################
% ###################################################
%
%                 PAPER HERE BELOW
%                              
% ###################################################
% ###################################################
% ###################################################



\subfile{01_intro.tex}

% ###################################################


\subfile{02_install.tex}

% ###################################################

\subfile{03_edipack2.tex}

% ###################################################

\subfile{04_cbinding.tex}

% ###################################################

\subfile{05_examples.tex}

% ###################################################


\subfile{06_conclusions_acknowledgement_appendix.tex}

% ###################################################
% ###################################################
% ###################################################


%% References with bibTeX database:
\bibliography{references}




\end{document}








