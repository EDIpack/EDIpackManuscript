\documentclass[dvipsnames]{SciPost}
\usepackage[bitstream-charter]{mathdesign}
\usepackage[english]{babel}
\usepackage[T1]{fontenc}
\usepackage{lmodern}
% \usepackage{amsmath}
% \usepackage{amssymb}
%\usepackage{amsfonts}
\usepackage{amsthm}
\usepackage{microtype}
\usepackage{ragged2e}
\usepackage{makecell}
\usepackage{array}
\usepackage{stackengine}
\newlength\llength
\llength=1.38ex\relax
\usepackage{makecell}
\usepackage{subfiles}
\usepackage{comment}
\usepackage{xspace}
\usepackage{array}
\usepackage{tabularx}
\usepackage{ltablex}
\usepackage[final]{showlabels} %inline: draft//final

% Prevent all line breaks in inline equations.
\binoppenalty=10000
\relpenalty=10000

  \hypersetup{
    breaklinks=true,
    colorlinks,
    pdfborder={0 0 0},
    linkcolor={red!50!black},
    citecolor={blue!50!black},
    urlcolor={blue!80!black}
  }

\urlstyle{same}

% Fix \cal and \mathcal characters look (so it's not the same as \mathscr)
\DeclareSymbolFont{usualmathcal}{OMS}{cmsy}{m}{n}
\DeclareSymbolFontAlphabet{\mathcal}{usualmathcal}
\DeclareMathAlphabet\mathbfcal{OMS}{cmsy}{b}{n}


\renewcommand{\texttt}[1]{%
  \begingroup
  \ttfamily
  \begingroup\lccode`~=`_\lowercase{\endgroup\def~}{/\discretionary{}{}{}}%
  \begingroup\lccode`~=`/\lowercase{\endgroup\def~}{/\discretionary{}{}{}}%
  \begingroup\lccode`~=`[\lowercase{\endgroup\def~}{[\discretionary{}{}{}}%
  \begingroup\lccode`~=`.\lowercase{\endgroup\def~}{.\discretionary{}{}{}}%
  \catcode`/=\active\catcode`[=\active\catcode`.=\active
  \scantokens{#1\noexpand}%
  \endgroup
}

% \setcounter{secnumdepth}{4} 
% \setcounter{tocdepth}{4}

\definecolor{mymauve}{rgb}{0.58, 0, 0.82}
%From ALF source:
\definecolor{comment-color}{rgb}{0.8,0.1,0.1}
\definecolor{keyword-color}{rgb}{0.3,0.3,1}





%LISTING PACKAGE AND CONFIG:
%%%%%%%%%%%%%%%%%%%%%%%%%%%%%%%%%%%%%%%%
\usepackage{listings}

% FORTRAN
%%%%%%%%%%%%%%%%%%%%%%%%%%%%%%%%%%%%%%%%
\lstdefinestyle{fstyle}{
  frame=lines,
  language={[03]Fortran},
  basicstyle={\ttfamily\footnotesize},
  stringstyle=\ttfamily,
  keywordstyle=\color{keyword-color},
  commentstyle=\color{comment-color}\footnotesize,
  morecomment=[l]{!\ },
  fontadjust=true,
  mathescape,
  %breakatwhitespace=false,
  keepspaces=true,
  showstringspaces=false,
  columns=fullflexible,
  frame=single,
  xleftmargin=3.4pt,
  xrightmargin=3.4pt,
  numbers=left, numberstyle=\tiny, stepnumber=1, numbersep=3pt
}



% PYTHON
%%%%%%%%%%%%%%%%%%%%%%%%%%%%%%%%%%%%%%%%
\lstdefinestyle{mypython}{
  frame=lines,
  language=python,
  basicstyle=\ttfamily\footnotesize,
  stringstyle=\ttfamily,
  commentstyle=\itshape,
  fontadjust=true,
  keywordstyle=\color{keyword-color},
  commentstyle=\color{comment-color}\footnotesize,
  morecomment=[l]{\#\ },
  mathescape,
  fontadjust=true,
  mathescape,
  % breakatwhitespace=false,
  keepspaces=true,
  showstringspaces=false,
  columns=fullflexible,
  frame=single,
  xleftmargin=3.4pt,
  xrightmargin=3.4pt,
  numbers=none, numberstyle=\tiny, stepnumber=1, numbersep=3pt
}


%%
%% Julia definition (c) 2014 Jubobs
%%
\lstdefinelanguage{Julia}%
  {morekeywords={abstract,break,case,catch,const,continue,do,else,elseif,%
      end,export,false,for,function,immutable,import,importall,if,in,%
      macro,module,otherwise,quote,return,switch,true,try,type,typealias,%
      using,while},%
   sensitive=true,%
   alsoother={\$},%
   morecomment=[l]\#,%
   morecomment=[n]{\#=}{=\#},%
   morestring=[s]{"}{"},%
   morestring=[m]{'}{'},%
}[keywords,comments,strings]%

\lstdefinestyle{myjulia}{
  frame=lines,
  language=julia,
  basicstyle=\ttfamily\footnotesize,
  stringstyle=\ttfamily,
  commentstyle=\itshape,
  fontadjust=true,
  keywordstyle=\color{keyword-color},
  commentstyle=\color{comment-color}\footnotesize,
  morecomment=[l]{\#\ },% Otherwise it's a comment only with space after #
  mathescape,
  fontadjust=true,
  mathescape,
  % breakatwhitespace=false,
  keepspaces=true,
  showstringspaces=false,
  columns=fullflexible,
  frame=single,
  xleftmargin=3.4pt,
  xrightmargin=3.4pt,
  numbers=none, numberstyle=\tiny, stepnumber=1, numbersep=3pt
}






% BASH
%%%%%%%%%%%%%%%%%%%%%%%%%%%%%%%%%%%%%%%%
\lstdefinestyle{mybash}{
  frame=lines,
  language=bash,
  % basicstyle=\footnotesize,
  basicstyle=\ttfamily,
  stringstyle=\color{mymauve},%\ttfamily,
  commentstyle=\color{comment-color},
  % keywordstyle=\bfseries,
  keywordstyle=\color{keyword-color},
  % It would be nice to have 'conda' here,
  % but then 'conda-forge' gets highlighted too.
  morekeywords={*,git,mkdir,cmake,make,pip,ninja,conda},
  captionpos=b,
  mathescape,
  fontadjust=true,
  breakatwhitespace=false,
  keepspaces=true,
  showstringspaces=false,
  columns=flexible,
  xleftmargin=3.4pt,
  xrightmargin=3.4pt,
  numbers=left, numberstyle=\tiny, stepnumber=1, numbersep=3pt
}




% C/C++
%%%%%%%%%%%%%%%%%%%%%%%%%%%%%%%%%%%%%%%%
\lstdefinestyle{cstyle}{
  frame=lines,
  language=C,
  % basicstyle=\footnotesize,
  basicstyle=\ttfamily,
  stringstyle=\color{mymauve},%\ttfamily,
  commentstyle=\color{comment-color},
  % keywordstyle=\bfseries,
  keywordstyle=\color{keyword-color},
  % It would be nice to have 'conda' here,
  % but then 'conda-forge' gets highlighted too.
  morekeywords={*,git,mkdir,cmake,make,pip,ninja,conda},
  captionpos=b,
  mathescape,
  fontadjust=true,
  breakatwhitespace=false,
  keepspaces=true,
  showstringspaces=false,
  columns=flexible,
  xleftmargin=3.4pt,
  xrightmargin=3.4pt,
  numbers=left, numberstyle=\tiny, stepnumber=1, numbersep=3pt
}









\renewcommand{\arraystretch}{1.4}
\newcolumntype{T}[1]{>{\tt\footnotesize\raggedright\arraybackslash}p{#1}}
\newcolumntype{D}[1]{>{\it\footnotesize\raggedright\arraybackslash}p{#1}}
\newcolumntype{M}[1]{>{\scriptsize\raggedright\arraybackslash}p{#1}}


\newcommand{\onlinecite}[1]{\cite{#1}}
% {\nocite{#1}\!\!\!\citenum{#1}} This was for CPC
\newcommand{\fixme}[1]{{\color{BrickRed} #1}}
\newcommand {\note}[1]{{\color{blue} [{\bf NOTE}: \bf #1]}}
\newcommand {\new}[1]{{\color{blue}\it #1}}




%Reference to a given labelled equation
%and definition of a bib. element.
%-------------------------------------------
\newcommand{\equ}[1]
{Eq.\,(\ref{#1})}

\newcommand{\figu}[1]
{Fig.\,\ref{#1}}

\newcommand{\secu}[1]
{Sec.\,\ref{#1}}

\newcommand{\ket}[1]
{|#1\rangle}

\newcommand{\bra}[1]
{\langle #1|}

\newcommand{\sgn}
{\mathop{\mathrm{sgn}}}



%SIMBOLI VARI
%%%%%%%%%%%%%%%%%%%%%%%%%%%%%%%%%%%%%%%%%%%%%%%%%%%%%%%
\def\a{\alpha}       \def\b{\beta}   \def\g{\gamma}   \def\d{\delta}
\def\e{\varepsilon}  \def\z{\zeta}   \def\h{\eta}     \def\th{\theta}
\def\k{\kappa}       \def\l{\lambda} \def\m{\mu}      \def\n{\nu}
\def\x{\xi}          \def\p{\pi}     \def\r{\rho}     \def\s{\sigma}
\def\t{\tau}         \def\f{\varphi} \def\ph{\varphi} \def\c{\chi}
\def\ps{\pi}        \def\y{\upsilon}\def\o{\omega}   \def\si{\varsigma}
\def\G{\Gamma}       \def\D{\Delta}  \def\Th{\Theta}  \def\L{\Lambda}
\def\X{\Xi}          \def\P{\Pi}     \def\Si{\Sigma}  \def\F{\Phi}
\def\Ps{\Psi}        \def\O{\Omega}  \def\Y{\Upsilon} \def\lg{\langle}

\def\PP{{\cal P}}\def\EE{{\cal E}}\def\MM{{\cal M}} \def\VV{{\cal V}}
\def\CC{{\cal C}}\def\FF{{\cal F}}\def\HH{{\cal H}}\def\WW{{\cal W}}
\def\TT{{\cal T}}\def\NN{{\cal N}}\def\BB{{\cal B}} \def\II{{\cal I}}
\def\RR{{\cal R}}\def\LL{{\cal L}}\def\JJ{{\cal J}} \def\OO{{\cal O}}
\def\DD{{\cal D}}
\def\AA{{\cal A}}
\def\GG{{\cal G}} \def\SS{{\cal S}}
\def\ZZ{{\cal Z}} \def\UU{{\cal U}}
\def\SB{{\cal S}{\cal B}}
\def\aa{{\V \a}}
\def\hh{{\V h}}\def\HHH{{\V H}}
%\def\AA{{\V A}}
%\def\GG{{\V G}}\def\BB{{\V B}}\def\aaa{{\V a}}\def\bbb{{\V b}}
\def\nn{{\V \n}}\def\pp{{\V p}}\def\mm{{\V m}}\def\qq{{\bf q}}
\def\RRR{\mathbb{R}} \def\CCC{\mathbb{C}} \def\NNN{\mathbb{N}}
\def\ZZZ{\mathbb{Z}}
%\def\TTT{\hbox{\msytw T}}



\def\ul{\underline}
\def\=={\equiv}
\def\defi{{\buildrel def \over =}}
\def\lft{\left} \def\rgt{\right} \def\dpr{\partial} \def\der{{\rm d}}
\def\us{\underline \s} \def\ue{{\underline \e}}
\def\la{\left\langle}
\def\ra{\right\rangle}
\def\qed{\raise1pt\hbox{\vrule height5pt width5pt depth0pt}}
\def\iome{i\omega_n} \def\iom{i\omega} \def\iom#1{i\omega_{#1}}
\def\iomn{i\omega_n}
\def\epsk{\epsilon({\bf k})} \def\Ga{\Gamma_{\alpha}}
\def\Seff{S_{eff}}  \def\dinf{$d\rightarrow\infty\,$}
\def\cG0{{\cal G}_0}
\def\cG{{\cal G}}  \def\cU{{\cal U}}  \def\cS{{\cal S}}
\def\spinup{\uparrow} \def\spindown{\downarrow} \def\spindw{\downarrow}
\def\up{\uparrow} \def\down{\downarrow} \def\dw{\downarrow}


\def\Ak{{\bf A}} \def\Akt{{\bf A}(t)} \def\Ek{{\mathbf E}}
% \def\Im{\mbox{Im}}
\def\=={\equiv}
\def\defi{{\buildrel def \over =}} \def\nt{\widetilde{n}}
\def\Im{{\rm Im}} \def\Re{{\rm Re}} \def\Tr{{\rm Tr}}
\def\det{{\rm det}\,} 


\def\ibra{\langle}
\def\iket{\rangle}

\def\ka{{\bf k}}
\def\vk{{\bf k}}
\def\qa{{\bf q}}
\def\vQ{{\bf Q}}
\def\vr{{\bf r}}
\def\q{{\bf q}}
\def\R{{\bf R}}
\def\vR{{\bf R}}
\def\kx{{ k_x}}
\def\ia{{\bf i}}
\def\ja{{\bf j}}

%\usepackage{bbold}
\def\11{\mathbb{1}}
\def\00{\mathbf{0}}
\def\NAME{{\rm EDIpack}\xspace}
\newcommand{\shortrightarrow}{\mathrel{\mkern3mu\rightarrow\mkern}}





%%%%%%%%%%%%%%%%%%%%%%%%%%%%%%%%%%%%%%%%%%%%%%%%%%%%%%% 
\fancypagestyle{SPstyle}{
\fancyhf{}
\lhead{\colorbox{scipostblue}{\bf \color{white} ~SciPost Physics Codebases }}
\rhead{{\bf \color{scipostdeepblue} ~Submission }}
\renewcommand{\headrulewidth}{1pt}
\fancyfoot[C]{\textbf{\thepage}}
}



\begin{document}

\pagestyle{SPstyle}

%TITLE
%%%%%%%%%% TODO: Write your article's title here
\begin{center}{
\Large \textbf{
\color{scipostdeepblue}{
EDIpack 2.0: A High-Performance, Interoperable Lanczos-based Solver for generic Quantum Impurity Models\\
}
}
}
\end{center}
%%%%%%%%%% END TODO: TITLE



%%%%%%%%%% AUTHORS
% Mark the corresponding author(s) with a superscript symbol in this order
% \star, \dagger, \ddagger, \circ, \S, \P, \parallel, ...
\begin{center}\textbf{
    Lorenzo Crippa\textsuperscript{1,2,3},
    Igor Krivenko\textsuperscript{1},
    Samuele Giuli\textsuperscript{4},
    Gabriele Bellomia\textsuperscript{4},
    Alexander Kowalski\textsuperscript{3},
    Francesco Petocchi\textsuperscript{5},
    Markus Wallerberger\textsuperscript{6},
    Giacomo Mazza\textsuperscript{7},
    Luca de Medici\textsuperscript{8},
    Giorgio Sangiovanni\textsuperscript{2,3},
    Massimo Capone\textsuperscript{4,9} and 
    Adriano Amaricci\textsuperscript{9$\star$}    
}\end{center}
%%%%%%%%%% END TODO: AUTHORS



%%%%%%%%%% TODO: AFFILIATIONS
% Write all affiliations here.
% Format: institute, city, country
\begin{center}
  \newcommand{\CNRIOM}{CNR-IOM, Istituto Officina dei Materiali,
  Consiglio Nazionale delle Ricerche, Via Bonomea 265, 34136
  Trieste, Italy}
\newcommand{\SISSA}{Scuola Internazionale Superiore di Studi Avanzati (SISSA),
  Via Bonomea 265, 34136 Trieste, Italy}
\newcommand{\ITPHamburg}{I. Institute of Theoretical Physics,
  University of Hamburg, Notkestrasse 9, 22607 Hamburg, Germany}
\newcommand{\WZBURG}{Institut f\"ur Theoretische Physik und
  Astrophysik,Universit\"at W\"urzburg, 97074 W\"urzburg, Germany}
\newcommand{\CTQMAT}{W\"urzburg-Dresden Cluster of Excellence ct.qmat, 01062 Dresden, Germany}
\newcommand{\Geneve}{Department of Quantum Matter Physics, University of
  Geneva, Quai Ernest-Ansermet 24, 1211 Geneva, Switzerland}
\newcommand{\UPISA}{Department of Physics ``E. Fermi'' University of
  Pisa, Largo B. Pontecorvo 3, 56127 Pisa, Italy}
\newcommand{\ESPCI}{LPEM, ESPCI Paris, PSL Research University, CNRS, Sorbonne Universit\'e, 75005 Paris, France}
\newcommand{\TUW}{Institure of Solid State Physics, TU Wien, 1040 Vienna, Austria}
{\bf 1} \ITPHamburg\\
{\bf 2} \CTQMAT\\
{\bf 3} \WZBURG\\
{\bf 4} \SISSA\\
{\bf 5} \Geneve\\
{\bf 6} \TUW\\   
{\bf 7} \UPISA\\
{\bf 8} \ESPCI\\
{\bf 9} \CNRIOM\\
\end{center}
%%%%%%%%%% END TODO: AFFILIATIONS



%%%%%%%%%% TODO: EMAIL
% Provide email address of corresponding author(s)
\begin{center}
\vspace{\baselineskip}
% $\star$ \href{mailto:email1}{\small email1}\,,\quad
$\star$ \href{mailto:edipack@cnr.iom.it}{\small edipack@cnr.iom.it}
\end{center}
%%%%%%%%%% END TODO: EMAIL



%%%%%%%%%% TODO: ABSTRACT
% Write your abstract here.
\section*{\color{scipostdeepblue}{Abstract}}
\fixme{Probably I can do a better or more evocative job here}
\textbf{\boldmath{%
EDIpack is a flexible, high-performance numerical library to solve generic quantum impurity problems using Lanczos-based exact diagonalization, as emerging from dynamical mean-field theory (DMFT) description of strongly correlated materials.
Its modular architecture provides Fortran APIs and bindings for C, interfaces with Python or Julia as well as with  TRIQS and W2Dynamics research platform. We discuss mathematical framework alongside implementation details and showcases illustrative applications.
The outlook includes further extensions for quantum materials, condensed matter physics, and quantum information applications.
}}
%%%%%%%%%% END TODO: ABSTRACT

\vspace{\baselineskip}


% %%%%%%%%%% BLOCK: Copyright information
% % This block will be filled during the proof stage, and finilized just before publication.
% % It exists here only as a placeholder, and should not be modified by authors.
% \noindent\textcolor{white!90!black}{%
% \fbox{\parbox{0.975\linewidth}{%
% \textcolor{white!40!black}{\begin{tabular}{lr}%
%   \begin{minipage}{0.6\textwidth}%
%     {\small Copyright attribution to authors. \newline
%     This work is a submission to SciPost Physics Codebases. \newline
%     License information to appear upon publication. \newline
%     Publication information to appear upon publication.}
%   \end{minipage} & \begin{minipage}{0.4\textwidth}
%     {\small Received Date \newline Accepted Date \newline Published Date}%
%   \end{minipage}
% \end{tabular}}
% }}
% }
% %%%%%%%%%% BLOCK: Copyright information


%%%%%%%%%% TODO: LINENO
% For convenience during refereeing we turn on line numbers:
\linenumbers
% You should run LaTeX twice in order for the line numbers to appear.
%%%%%%%%%% END TODO: LINENO

%%%%%%%%%% TODO: TOC 
% Guideline: if your paper is longer that 6 pages, include a TOC
% To remove the TOC, simply cut the following block
\vspace{10pt}
\noindent\rule{\textwidth}{1pt}
\tableofcontents
\noindent\rule{\textwidth}{1pt}
\vspace{10pt}
%%%%%%%%%% END TODO: TOC





% ###################################################
% ###################################################
% ###################################################
%
%                 PAPER HERE BELOW
%                              
% ###################################################
% ###################################################
% ###################################################



\subfile{01_intro.tex}

% ###################################################


\subfile{02_install.tex}

% ###################################################

\subfile{03_edipack2.tex}

% ###################################################

\subfile{04_cbinding.tex}

% ###################################################

\subfile{05_examples.tex}

% ###################################################


\subfile{06_conclusions_acknowledgement_appendix.tex}

% ###################################################
% ###################################################
% ###################################################


%% References with bibTeX database:
\bibliography{references}




\end{document}








