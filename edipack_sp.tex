\documentclass[dvipsnames]{SciPost}


%%%%%%%%%%%%%%%%%%%%%%%%%%%%%%%%%%%%%%%%%%%%%%%%%%%%%%%

\usepackage[english]{babel}
\usepackage[T1]{fontenc}
\usepackage{lmodern}
\usepackage{amsmath,amssymb,amsfonts,amsthm}
\usepackage{makecell}
\usepackage{array}
\usepackage{stackengine}
\newlength\llength
\llength=1.38ex\relax
% \usepackage[dvipsnames]{xcolor}
\usepackage{makecell}
\usepackage{subfiles}
\usepackage{comment}

\usepackage{xspace}

% Prevent all line breaks in inline equations.
\binoppenalty=10000
\relpenalty=10000

  \hypersetup{
    breaklinks=true,
    colorlinks,
    pdfborder={0 0 0},
    linkcolor={red!50!black},
    citecolor={blue!50!black},
    urlcolor={blue!80!black}
  }

\usepackage[bitstream-charter]{mathdesign}
\urlstyle{same}

% Fix \cal and \mathcal characters look (so it's not the same as \mathscr)
\DeclareSymbolFont{usualmathcal}{OMS}{cmsy}{m}{n}
\DeclareSymbolFontAlphabet{\mathcal}{usualmathcal}
\DeclareMathAlphabet\mathbfcal{OMS}{cmsy}{b}{n}


\renewcommand{\texttt}[1]{%
  \begingroup
  \ttfamily
  \begingroup\lccode`~=`_\lowercase{\endgroup\def~}{/\discretionary{}{}{}}%
  \begingroup\lccode`~=`/\lowercase{\endgroup\def~}{/\discretionary{}{}{}}%
  \begingroup\lccode`~=`[\lowercase{\endgroup\def~}{[\discretionary{}{}{}}%
  \begingroup\lccode`~=`.\lowercase{\endgroup\def~}{.\discretionary{}{}{}}%
  \catcode`/=\active\catcode`[=\active\catcode`.=\active
  \scantokens{#1\noexpand}%
  \endgroup
}

% \setcounter{secnumdepth}{4} 
% \setcounter{tocdepth}{4}

\definecolor{mymauve}{rgb}{0.58, 0, 0.82}
%From ALF source:
\definecolor{comment-color}{rgb}{0.8,0.1,0.1}
\definecolor{keyword-color}{rgb}{0.3,0.3,1}

\usepackage{listings}


\lstdefinestyle{fstyle}{
  frame=lines,
  language={[03]Fortran},
  basicstyle={\ttfamily\footnotesize},
  stringstyle=\ttfamily,
  keywordstyle=\color{keyword-color},
  commentstyle=\color{comment-color}\footnotesize,
  morecomment=[l]{!\ },% Otherwise it's a comment only with space after !
  fontadjust=true,
  mathescape,
  % breakatwhitespace=false,
  keepspaces=true,
  showstringspaces=false,
  columns=fullflexible,
  frame=single,
  numbers=left, numberstyle=\tiny, stepnumber=1, numbersep=3pt
}



\lstdefinestyle{mypython}{
  frame=lines,
  language=python,
  basicstyle=\ttfamily\footnotesize,
  stringstyle=\ttfamily,
  commentstyle=\itshape,
  fontadjust=true,
  keywordstyle=\color{keyword-color},
  commentstyle=\color{comment-color}\footnotesize,
  morecomment=[l]{\#\ },% Otherwise it's a comment only with space after #
  mathescape,
  fontadjust=true,
  mathescape,
  % breakatwhitespace=false,
  keepspaces=true,
  showstringspaces=false,
  columns=fullflexible,
  frame=single,
  numbers=left, numberstyle=\tiny, stepnumber=1, numbersep=3pt
}


\lstdefinestyle{mybash}{
  frame=lines,
  language=bash,
  % basicstyle=\footnotesize,
  basicstyle=\ttfamily,
  stringstyle=\color{mymauve},%\ttfamily,
  % commentstyle=\itshape,
  commentstyle=\color{comment-color},
  morecomment=[l]{\#\ }% Comment only with space after #
  % keywordstyle=\bfseries,
  keywordstyle=\color{keyword-color},
  % It would be nice to have 'conda' here,
  % but then 'conda-forge' gets highlighted too.
  morekeywords={*,git,mkdir,cmake,make,pip},
  captionpos=b,
  mathescape,
  fontadjust=true,
  % breakatwhitespace=false,
  keepspaces=true,
  showstringspaces=false,
  columns=flexible,
  xleftmargin=3.4pt,
  xrightmargin=3.4pt,
  numbers=left, numberstyle=\tiny, stepnumber=1, numbersep=3pt
}


\lstdefinestyle{cstyle}{
  frame=lines,
  language=c,
  basicstyle=\footnotesize,
  stringstyle=\ttfamily,
  commentstyle=\itshape,
  fontadjust=true,
  keywordstyle=\color{red},
  % morekeywords={*,...},
  mathescape,
  numbers=left, numberstyle=\tiny, stepnumber=1, numbersep=3pt
}

% % MINTED:
% % \usepackage[cache=true,cachedir=minted-cache]{minted}
% % \usepackage[finalizecache=true,cachedir=minted-cache]{minted}
% \usepackage[frozencache=true,cachedir=minted-cache]{minted}
% \setminted[fortran]{linenos,mathescape,frame=lines,style=bw,framesep=1mm,
%   baselinestretch=1,fontsize=\footnotesize}



\usepackage{array}
\usepackage{tabularx}
\usepackage{ltablex}



\renewcommand{\arraystretch}{1.4}
\newcolumntype{T}[1]{>{\tt\footnotesize\raggedright\arraybackslash}p{#1}}
\newcolumntype{D}[1]{>{\it\footnotesize\raggedright\arraybackslash}p{#1}}
\newcolumntype{M}[1]{>{\scriptsize\raggedright\arraybackslash}p{#1}}


\newcommand{\onlinecite}[1]{\cite{#1}}
% {\nocite{#1}\!\!\!\citenum{#1}} This was for CPC
\newcommand{\fixme}[1]{{\color{BrickRed} #1}}
\newcommand {\note}[1]{{\color{blue} [{\bf NOTE}: \bf #1]}}
\newcommand {\aac}[1]{{\color{red} [{\bf AA}: \bf #1]}}
\newcommand {\new}[1]{{\color{blue}\it #1}}

\usepackage[inline]{showlabels}


%Reference to a given labelled equation
%and definition of a bib. element.
%-------------------------------------------
\newcommand{\equ}[1]
{Eq.\,(\ref{#1})}

\newcommand{\figu}[1]
{Fig.\,\ref{#1}}

\newcommand{\secu}[1]
{Sec.\,\ref{#1}}

\newcommand{\ket}[1]
{|#1\rangle}

\newcommand{\bra}[1]
{\langle #1|}

\newcommand{\sgn}
{\mathop{\mathrm{sgn}}}



%SIMBOLI VARI
%%%%%%%%%%%%%%%%%%%%%%%%%%%%%%%%%%%%%%%%%%%%%%%%%%%%%%%
\def\a{\alpha}       \def\b{\beta}   \def\g{\gamma}   \def\d{\delta}
\def\e{\varepsilon}  \def\z{\zeta}   \def\h{\eta}     \def\th{\theta}
\def\k{\kappa}       \def\l{\lambda} \def\m{\mu}      \def\n{\nu}
\def\x{\xi}          \def\p{\pi}     \def\r{\rho}     \def\s{\sigma}
\def\t{\tau}         \def\f{\varphi} \def\ph{\varphi} \def\c{\chi}
\def\ps{\pi}        \def\y{\upsilon}\def\o{\omega}   \def\si{\varsigma}
\def\G{\Gamma}       \def\D{\Delta}  \def\Th{\Theta}  \def\L{\Lambda}
\def\X{\Xi}          \def\P{\Pi}     \def\Si{\Sigma}  \def\F{\Phi}
\def\Ps{\Psi}        \def\O{\Omega}  \def\Y{\Upsilon} \def\lg{\langle}

\def\PP{{\cal P}}\def\EE{{\cal E}}\def\MM{{\cal M}} \def\VV{{\cal V}}
\def\CC{{\cal C}}\def\FF{{\cal F}}\def\HH{{\cal H}}\def\WW{{\cal W}}
\def\TT{{\cal T}}\def\NN{{\cal N}}\def\BB{{\cal B}} \def\II{{\cal I}}
\def\RR{{\cal R}}\def\LL{{\cal L}}\def\JJ{{\cal J}} \def\OO{{\cal O}}
\def\DD{{\cal D}}
\def\AA{{\cal A}}
\def\GG{{\cal G}} \def\SS{{\cal S}}
\def\ZZ{{\cal Z}} \def\UU{{\cal U}}
\def\SB{{\cal S}{\cal B}}
\def\aa{{\V \a}}
\def\hh{{\V h}}\def\HHH{{\V H}}
%\def\AA{{\V A}}
%\def\GG{{\V G}}\def\BB{{\V B}}\def\aaa{{\V a}}\def\bbb{{\V b}}
\def\nn{{\V \n}}\def\pp{{\V p}}\def\mm{{\V m}}\def\qq{{\bf q}}
\def\RRR{\mathbb{R}} \def\CCC{\mathbb{C}} \def\NNN{\mathbb{N}}
\def\ZZZ{\mathbb{Z}}
%\def\TTT{\hbox{\msytw T}}



\def\ul{\underline}
\def\=={\equiv}
\def\defi{{\buildrel def \over =}}
\def\lft{\left} \def\rgt{\right} \def\dpr{\partial} \def\der{{\rm d}}
\def\us{\underline \s} \def\ue{{\underline \e}}
\def\la{\left\langle}
\def\ra{\right\rangle}
\def\qed{\raise1pt\hbox{\vrule height5pt width5pt depth0pt}}
\def\iome{i\omega_n} \def\iom{i\omega} \def\iom#1{i\omega_{#1}}
\def\iomn{i\omega_n}
\def\epsk{\epsilon({\bf k})} \def\Ga{\Gamma_{\alpha}}
\def\Seff{S_{eff}}  \def\dinf{$d\rightarrow\infty\,$}
\def\cG0{{\cal G}_0}
\def\cG{{\cal G}}  \def\cU{{\cal U}}  \def\cS{{\cal S}}
\def\spinup{\uparrow} \def\spindown{\downarrow} \def\spindw{\downarrow}
\def\up{\uparrow} \def\down{\downarrow} \def\dw{\downarrow}


\def\Ak{{\bf A}} \def\Akt{{\bf A}(t)} \def\Ek{{\mathbf E}}
% \def\Im{\mbox{Im}}
\def\=={\equiv}
\def\defi{{\buildrel def \over =}} \def\nt{\widetilde{n}}
\def\Im{{\rm Im}} \def\Re{{\rm Re}} \def\Tr{{\rm Tr}}
\def\det{{\rm det}\,} 


\def\ibra{\langle}
\def\iket{\rangle}

\def\ka{{\bf k}}
\def\vk{{\bf k}}
\def\qa{{\bf q}}
\def\vQ{{\bf Q}}
\def\vr{{\bf r}}
\def\q{{\bf q}}
\def\R{{\bf R}}
\def\vR{{\bf R}}
\def\kx{{ k_x}}
\def\ia{{\bf i}}
\def\ja{{\bf j}}

\usepackage{bbold}
\def\11{\mathbb{1}}
\def\00{\mathbf{0}}
\def\NAME{{\rm EDIpack}\xspace}
\newcommand{\shortrightarrow}{\mathrel{\mkern3mu\rightarrow\mkern}}





%%%%%%%%%%%%%%%%%%%%%%%%%%%%%%%%%%%%%%%%%%%%%%%%%%%%%%% 
\fancypagestyle{SPstyle}{
\fancyhf{}
\lhead{\colorbox{scipostblue}{\bf \color{white} ~SciPost Physics Codebases }}
\rhead{{\bf \color{scipostdeepblue} ~Submission }}
\renewcommand{\headrulewidth}{1pt}
\fancyfoot[C]{\textbf{\thepage}}
}



\begin{document}


\pagestyle{SPstyle}

%TITLE
\begin{center}{\Large \textbf{\color{scipostdeepblue}{
%%%%%%%%%% TODO: Write your article's title here
EDIpack: A generic and interoperable Lanczos-based solver for quantum impurity problems\\
%%%%%%%%%% END TODO: TITLE
}}}\end{center}


\begin{center}\textbf{
%%%%%%%%%% TODO: AUTHORS
% Write the author list here. 
% Use (full) first name (+ middle name initials) + surname format.
% Separate subsequent authors by a comma, omit comma and use "and" for the last author.
% Mark the corresponding author(s) with a superscript symbol in this order
% \star, \dagger, \ddagger, \circ, \S, \P, \parallel, ...
    Lorenzo Crippa\textsuperscript{1,2,3$\star$},
    Igor Krivenko\textsuperscript{1$\star$},
    Samuele Giuli\textsuperscript{4$\star$},
    Gabriele Bellomia\textsuperscript{4$\star$},
    Alexander Kowalski\textsuperscript{3},
    Francesco Petocchi\textsuperscript{5},
    Markus Wallerberger\textsuperscript{6},
    Giacomo Mazza\textsuperscript{7},
    Luca de Medici\textsuperscript{8},
    Giorgio Sangiovanni\textsuperscript{2,3},
    Massimo Capone\textsuperscript{4,9$\star$} and 
    Adriano Amaricci\textsuperscript{9$\star$}    
%%%%%%%%%% END TODO: AUTHORS
}\end{center}

\begin{center}
%%%%%%%%%% TODO: AFFILIATIONS
% Write all affiliations here.
% Format: institute, city, country
  \newcommand{\CNRIOM}{CNR-IOM, Istituto Officina dei Materiali,
  Consiglio Nazionale delle Ricerche, Via Bonomea 265, 34136
  Trieste, Italy}
\newcommand{\SISSA}{Scuola Internazionale Superiore di Studi Avanzati (SISSA),
  Via Bonomea 265, 34136 Trieste, Italy}
\newcommand{\ITPHamburg}{I. Institute of Theoretical Physics,
  University of Hamburg, Notkestrasse 9, 22607 Hamburg, Germany}
\newcommand{\WZBURG}{Institut f\"ur Theoretische Physik und
  Astrophysik,Universit\"at W\"urzburg, 97074 W\"urzburg, Germany}
\newcommand{\CTQMAT}{W\"urzburg-Dresden Cluster of Excellence ct.qmat, 01062 Dresden, Germany}
\newcommand{\Geneve}{Department of Quantum Matter Physics, University of
  Geneva, Quai Ernest-Ansermet 24, 1211 Geneva, Switzerland}
\newcommand{\UPISA}{Department of Physics ``E. Fermi'' University of
  Pisa, Largo B. Pontecorvo 3, 56127 Pisa, Italy}
\newcommand{\ESPCI}{LPEM, ESPCI Paris, PSL Research University, CNRS, Sorbonne Universit\'e, 75005 Paris, France}
\newcommand{\TUW}{Institure of Solid State Physics, TU Wien, 1040 Vienna, Austria}

{\bf 1} \ITPHamburg\\
{\bf 2} \CTQMAT\\
{\bf 3} \WZBURG\\
{\bf 4} \SISSA\\
{\bf 5} \Geneve\\
{\bf 6} \TUW\\   
{\bf 7} \UPISA\\
{\bf 8} \ESPCI\\
{\bf 9} \CNRIOM\\
%%%%%%%%%% END TODO: AFFILIATIONS
%%%%%%%%%% TODO: EMAIL
% Provide email address of corresponding author(s)
\vspace{\baselineskip}
$\star$ \href{mailto:email1}{\small email1}\,,\quad
%%%%%%%%%% END TODO: EMAIL
\end{center}

\section*{\color{scipostdeepblue}{Abstract}}
\textbf{\boldmath{%
%%%%%%%%%% TODO: ABSTRACT
% Write your abstract here.
The abstract is in boldface, and should fit in 8 lines. It should be written in a clear and accessible style, emphasizing the context, the problem(s) studied, the methods used, the results obtained, the conclusions reached, and the outlook. You can add a table contents, recommended if your paper is more than 6 pages long.
%%%%%%%%%% END TODO: ABSTRACT
}}

\vspace{\baselineskip}


%%%%%%%%%% BLOCK: Copyright information
% This block will be filled during the proof stage, and finilized just before publication.
% It exists here only as a placeholder, and should not be modified by authors.
\noindent\textcolor{white!90!black}{%
\fbox{\parbox{0.975\linewidth}{%
\textcolor{white!40!black}{\begin{tabular}{lr}%
  \begin{minipage}{0.6\textwidth}%
    {\small Copyright attribution to authors. \newline
    This work is a submission to SciPost Physics Codebases. \newline
    License information to appear upon publication. \newline
    Publication information to appear upon publication.}
  \end{minipage} & \begin{minipage}{0.4\textwidth}
    {\small Received Date \newline Accepted Date \newline Published Date}%
  \end{minipage}
\end{tabular}}
}}
}
%%%%%%%%%% BLOCK: Copyright information


%%%%%%%%%% TODO: LINENO
% For convenience during refereeing we turn on line numbers:
\linenumbers
% You should run LaTeX twice in order for the line numbers to appear.
%%%%%%%%%% END TODO: LINENO

%%%%%%%%%% TODO: TOC 
% Guideline: if your paper is longer that 6 pages, include a TOC
% To remove the TOC, simply cut the following block
\vspace{10pt}
\noindent\rule{\textwidth}{1pt}
\tableofcontents
\noindent\rule{\textwidth}{1pt}
\vspace{10pt}
%%%%%%%%%% END TODO: TOC





% ##################################################################
% ##################################################################
% ##################################################################
%
%                              PAPER HERE BELOW
%                              
% ##################################################################
% ##################################################################
% ##################################################################



\subfile{01_intro.tex}

% ##################################################################


\subfile{02_install.tex}

% ##################################################################

\subfile{03_edipack2.tex}

% ##################################################################

\subfile{04_cbinding.tex}

% ##################################################################

\subfile{05_examples.tex}

% ##################################################################


\subfile{06_conclusions_acknowledgement_appendix.tex}

% ##################################################################
% ##################################################################
% ##################################################################




%% References with bibTeX database:
\bibliography{references}






\end{document}








